% To compile in APA format: pdflatex "\def\apaformat{1}% To compile in APA format: pdflatex "\def\apaformat{1}% To compile in APA format: pdflatex "\def\apaformat{1}% To compile in APA format: pdflatex "\def\apaformat{1}\input{main}"
% To compile in original format: pdflatex main.tex

% \def\apaformat{1}

\ifdefined\apaformat
  \input{cidlab/draft}
\else
  \input{cidlab/pretty}
\fi
% Main document content starts here


\input{includes/notes}

% Comment this out to remove notes:
\shownotestrue

\input{includes/macros}

\usepackage{algorithm}
\usepackage{algpseudocode}
\usepackage{listings}
\lstset{extendedchars=true,upquote=true}
% Handle UTF-8 characters in listings using literate option
\lstdefinestyle{utf8style}{
  extendedchars=true,
  upquote=true,
  literate={≈}{{$\approx$}}1
}
\lstset{style=utf8style}

\renewcommand{\subsubsection}[1]{\textit{#1}.}

\begin{document}

\maketitle

\introduction{}

\noindent{}%
Cognitive models serve dual purposes in psychological research. On one hand, they can be used to understand the cognitive infrastructure---the underlying mechanisms and processes that give rise to behavior. On the other hand, cognitive models can function as measurement tools, translating behavioral observations into interpretable psychological parameters. This paper focuses on the latter: we develop computationally efficient methods for uncertainty quantification in cognitive models used as measurement instruments. The need for computational efficiency becomes critical in large-scale applications such as meta-analyses across hundreds of studies, real-time experimental analysis during data collection, and individual differences research requiring parameter estimation for thousands of participants. Traditional approaches to uncertainty quantification, which rely on computationally expensive resampling or Bayesian inference, create barriers in these contexts. We propose a bootstrap method that exploits the synthetic likelihood structure of a class of cognitive models we call `EZ' models, achieving dramatic speed improvements while maintaining statistical validity.

\section{Synthetic likelihood methods}

Synthetic likelihood methods provide a framework for statistical inference when the likelihood function is intractable but summary statistics with known or estimable distributions are available \cite{wood_statistical_2010,Price2017}. The core idea is to replace the full likelihood with a likelihood based on summary statistics, which can be computed efficiently even when the full data likelihood cannot.

The synthetic likelihood approach was introduced by \citeauthor{wood_statistical_2010} for inference in complex ecological systems with chaotic dynamics. When direct likelihood evaluation is impossible, one can extract phase-insensitive summary statistics from the data and compare them to summary statistics simulated from the model. This approach has been extended to Bayesian settings \cite{Price2017}, where the synthetic likelihood is used within a Bayesian framework to enable posterior inference.

Certain cognitive models, which we will refer to here as the `EZ' models (after the EZ diffusion model; \citeNP{WagenmakersEtAl2007}), similarly fit the description of synthetic likelihood models, and provide a natural application of methods specific to that class. For example, in the EZ diffusion model, rather than working with the full likelihood over all trial-level data, we work with three summary statistics: accuracy rate, mean response time, and response time variance. These statistics have known sampling distributions, allowing us to construct a likelihood function directly from the summaries \cite{chávez_de_la_peña_vandekerckhove:in_press:hierarchical}. This synthetic likelihood approach is particularly powerful because the EZ model provides analytical expressions that map parameters to these summary statistics, eliminating the need for simulation, and paving the way for uncertainty quantification using a variation on a classic transformation-of-variables technique, which we will describe after introducing the model class.

\section{The class of EZ models}

We will rely on a class of cognitive models we call `EZ'.  EZ models are defined by the existence of an invertible system of equations that maps a model parameter vector $\theta$ to a set of summary statistics $S$.  The vector $\theta$ need not contain \textit{all} of the model's parameters, but may be limited to a subset of interest (as in the application by \citeNP{wood_statistical_2010}).  The system consists of a set of smooth `forward' equations $f$ and `inverse' equations $i$:
\begin{eqnarray}
    f& :&  \theta \mapsto S \nonumber\\
    i& :&  S \mapsto \theta\label{eq:generic:inv}
\end{eqnarray}

The EZ class contains some well-known cognitive models.  We take signal detection theory (SDT; \citeNP{swets_green:1966:signal}) as an example:
\begin{eqnarray}
    (\HitRate,\FalseAlarm) &=& f_\text{SDT}(\dprime,\criterion) \nonumber \\
    &\Leftrightarrow& \left\{
    \begin{array}{rcl}
         \HitRate&=& \Phi(\dprime - \criterion) \\
         \FalseAlarm&=& \Phi(-\criterion) \\
    \end{array}\right.
\end{eqnarray}
where $\Phi$ denotes the standard normal cumulative distribution function. The inverse equations map summary statistics back to parameters:
\begin{eqnarray}
    (\dprime,\criterion) &=& i_\text{SDT}(\HitRate,\FalseAlarm) \nonumber \\
    &\Leftrightarrow& \left\{
    \begin{array}{rcl}
         \dprime&=& \Phi^{-1}(\HitRate) - \Phi^{-1}(\FalseAlarm) \\
         \criterion&=& -\Phi^{-1}(\FalseAlarm) \\
    \end{array}\right.
\end{eqnarray}
where $\Phi^{-1}$ denotes the inverse standard normal cumulative distribution function. Here, $\HitRate$ denotes the hit rate (probability of correctly identifying a signal), $\FalseAlarm$ denotes the false alarm rate (probability of incorrectly identifying a signal when none is present), $\dprime$ denotes sensitivity (the ability to discriminate signal from noise), and $\criterion$ denotes the criterion (the decision threshold).

Other members of the EZ class include the EZ diffusion model (\citeNP{WagenmakersEtAl2007}), the EZ circular diffusion model \cite{QarehdaghiRad2024}, and some simple multinomial processing tree models with analytical solutions. {These models are discussed in detail below.}


\section{The EZ bootstrap procedure}\label{sec:ez_bootstrap}

We propose a bootstrap method we call `EZ bootstrap' that exploits the synthetic likelihood structure of EZ models. The method operates directly on summary statistics, using their known sampling distributions to perform a parametric bootstrap. 

In the abstract, EZ bootstrap proceeds along the following steps. First, (1) compute summary statistics $S$ from the {observed} data. Then, (2) for each bootstrap repetition $b = 1, \ldots, B$: (2a) sample summary statistics $S^*_b$ from their known sampling distributions $\mathcal{D}_S$ (the requirement that these distributions be known or estimable is fundamental to the method), and (2b) transform $S^*_b$ through the inverse equations $i$ to obtain bootstrap parameter estimates $\theta^*_b = i(S^*_b)$. Finally, (3) aggregate the bootstrap parameter estimates $\{\theta^*_1, \ldots, \theta^*_B\}$ to compute means, standard deviations, and credible intervals.

This procedure is formalized in Algorithm~\ref{alg:ez_bootstrap}. EZ bootstrap is fundamentally a transformation-of-variables approach to uncertainty estimation. Rather than resampling raw data and re-estimating parameters, we resample the summary statistics from their known distributions, then transform these resampled statistics through the inverse equations to obtain parameter estimates. {Specifically, in Algorithm~\ref{alg:ez_bootstrap} we sample $S^*_b$ from $\mathcal{D}_S$ for each bootstrap repetition $b = 1, \ldots, B$, then compute $\theta^*_b = i(S^*_b)$ to obtain the bootstrap parameter estimates $\{\theta^*_1, \ldots, \theta^*_B\}$.} This transformation preserves the statistical properties of the bootstrap while eliminating the computational cost of data resampling. 

\subsection{Transformation of variables for uncertainty quantification}
Transformation of variables is a fundamental method in statistics for propagating uncertainty through deterministic transformations (\citeNP{rice2006mathematical}, see also \citeNP{hayes2013relative}). When a random variable $X$ with known distribution is transformed through a function $Y = g(X)$, the distribution of $Y$ can be derived using the change-of-variables formula. This technique is central to the EZ bootstrap method.

In the context of uncertainty quantification, transformation of variables allows us to derive the distribution of parameter estimates from the distribution of summary statistics in EZ models. The inverse equations (Eq.~\ref{eq:generic:inv}) provide a deterministic transformation from summary statistics to parameters. By applying this transformation to bootstrap samples of the summary statistics, we obtain bootstrap samples of the parameters, which can then be used to quantify uncertainty.

The key advantage of this approach is computational efficiency. Rather than resampling raw data and re-estimating parameters for each bootstrap sample---which would require expensive optimization or simulation \cite{efron1994introduction}---we resample the summary statistics from their known distributions and apply the analytical transformation, {as formalized in Algorithm~\ref{alg:ez_bootstrap}}. This eliminates the computational bottleneck while preserving the statistical properties of the bootstrap.

The transformation-of-variables approach is made possible in EZ models particularly because the inverse transformation is analytical, smooth, and one-to-one (within the parameter space, or a relevant subspace thereof). 

\begin{algorithm}[bt]
\caption{EZ Bootstrap Procedure}
\label{alg:ez_bootstrap}
\begin{algorithmic}[1]
\Require Observed summary statistics $S$, number of bootstrap samples $B$, inverse transformation $i: S \mapsto \theta$, sampling distributions $\mathcal{D}_S$ for each component of $S$
\Ensure Bootstrap parameter estimates $\{\theta^*_1, \ldots, \theta^*_B\}$
\State $\Theta^* \leftarrow \emptyset$
\For{$b = 1$ to $B$}
    \State Sample $S^*_b \sim \mathcal{D}_S$ from known sampling distributions of $S$
    \State $\theta^*_b \leftarrow i(S^*_b)$
    \State $\Theta^* \leftarrow \Theta^* \cup \{\theta^*_b\}$
\EndFor
\Return $\Theta^*$
\end{algorithmic}
\end{algorithm}

\subsection{Theoretical justification for coverage preservation} The bootstrap procedure maintains proper coverage properties when the inverse transformation $i: S \mapsto \theta$
is a continuous, differentiable function. Under the bootstrap sampling scheme, we generate $S^*$ from its known distribution, then compute $\theta^* = i(S^*)$. Since $i$ is deterministic and the bootstrap samples $S^*$ have the correct sampling distributions, the transformed samples $\theta^*$ correctly represent the sampling distribution of the parameter estimates. This follows from the invariance property of bootstrap methods under smooth transformations \cite{rice2006mathematical}. The coverage properties are preserved because the bootstrap distribution of the transformed parameters matches the true sampling distribution of the parameter estimates -- up to the accuracy of the assumed distributions for the summary statistics.

\section{Design matrix extension}

In applied contexts, we often constrain parameters across conditions. These constraints can be expressed using design matrices. For example, {for diffusion models,} if drift rates, {$\drift$,} vary linearly with a stimulus property $x_i$, we can write:
$$\drift_i = \beta_0 + \beta_1 x_i.$$
If they vary quarilinearly, then:
$$\drift_i = \beta_0 + \beta_1 x_i + \beta_2 x_i^2.$$

This generalizes, so that parameter constraints can be expressed as:
$$\Delta = X \times \boldsymbol{\beta}$$
where $\Delta$ is a vector of parameters, $X$ is a design matrix, and $\boldsymbol{\beta}$ is a vector of regression weights.

If the model parameters $\Delta$ are known and the model is identified, the weights $\boldsymbol{\beta}$ can be derived via matrix inversion:
\begin{equation}
    \hat{\boldsymbol{\beta}} = X^{-1} \times \Delta \label{eq:design:solution}
\end{equation}

Importantly, adding the design matrix extension \textit{keeps the model inside the EZ class} and does not change the bootstrap logic.  This is easy to prove: Let $i: S \mapsto \theta$ denote the EZ inverse transformation mapping summary statistics $S$ to parameters $\theta$ (Eq.~\ref{eq:generic:inv}), and let $h: \theta \mapsto \boldsymbol{\beta}$ denote the design matrix transformation mapping parameters $\theta$ to beta weights $\boldsymbol{\beta}$ via $h(\theta) = X^{-1}\theta$ (where $X$ is the design matrix). The basic EZ bootstrap generates bootstrap samples $\theta^* = i(S^*)$ where $S^*$ is resampled from the known distribution of summary statistics. With the design matrix extension, we generate bootstrap samples $\boldsymbol{\beta}^* = h(i(S^*)) = X^{-1}i(S^*)$. 
Since $i$ is a deterministic, smooth transformation (by the definition of EZ models) and $h$ is a deterministic linear transformation (matrix multiplication), their composition $\left(h \circ i\right): S \mapsto \boldsymbol{\beta}$ is also deterministic and smooth. By the same invariance property of bootstrap methods under smooth transformations \cite{rice2006mathematical}, if the bootstrap distribution of $S^*$ correctly represents the sampling distribution of $S$, then the bootstrap distribution of $\boldsymbol{\beta}^* = (h \circ i)(S^*)$ correctly represents the sampling distribution of $\boldsymbol{\beta} = (h \circ i)(S)$. Therefore, the design matrix extension preserves the bootstrap's coverage properties: the model remains in the EZ class because we still use the EZ inverse transformation, and the bootstrap logic is unchanged because we are still propagating uncertainty through a deterministic transformation.

\section{Application to the simple diffusion model}

The drift diffusion model (DDM) provides a unified account of decision-making processes, modeling both response accuracy and response time as the result of an evidence accumulation process \cite{Ratcliff1978,Wagenmakers2009Bertelson}. This framework has proven valuable in cognitive psychometrics, enabling the translation of behavioral patterns into interpretable cognitive parameters \cite<e.g.,>{pe_etal:2013:relationship,schubert_etal:2019:neuroscience,vandekerckhove:2014:simultaneous}.

The EZ diffusion model \cite{WagenmakersEtAl2007} was originally proposed as a method to simplify use of the DDM by providing analytical expressions that map three summary statistics---accuracy rate, mean response time, and response time variance---to three cognitive parameters: drift rate, boundary separation, and non-decision time. This makes the EZ framework a synthetic likelihood model: it provides a likelihood function based on summary statistics rather than raw trial data.

\subsection{The EZ diffusion model}

The EZ diffusion model provides analytical expressions for three summary statistics from a set of $n$ two-choice decision trials.  Let $\acc$ denote the observed accuracy rate, $\meanrt$ the observed mean response time, and $\varrt$ the observed response time variance. The model's three parameters are the drift rate $\drift$, boundary separation $\boundarysep$, and non-decision time $\ndt$.
This allows us to define forward equations 
$
f_\text{DIF}:
\left(\drift,\boundarysep,\ndt\right)
\mapsto
\left(\acc, \meanrt, \varrt\right)
$ and inverse equations 
$
i_\text{DIF}:
\left(\acc, \meanrt, \varrt\right)
\mapsto
\left(\drift,\boundarysep,\ndt\right)
$.

The EZ forward equations map parameters to summary statistics. Let $y = \exp(-\boundarysep \drift)$ where $\boundarysep$ is boundary separation and $\drift$ is drift rate. Then,
\begin{eqnarray}
\acc &=& \frac{1}{1+y} \nonumber\\
\meanrt &=& \ndt + \left(\frac{\boundarysep}{2\drift}\right) \left(\frac{1-y}{1+y}\right) \nonumber\\
\varrt &=& \left(\frac{\boundarysep}{2\drift^3}\right) \left(\frac{1-2\boundarysep\drift y - y^2}{(1+y)^2}\right) \nonumber
\end{eqnarray}
When $\drift = 0$ (drift rate is zero), the forward equations simplify to $\acc = 0.5$, $\meanrt = \ndt + \boundarysep$, and $\varrt = \boundarysep^2$.

The EZ inverse equations map summary statistics to parameters. Let $y = \log(\acc/(1-\acc))$ be the logit of accuracy. Then,
\begin{eqnarray}
\drift &=& \text{sign}\left(\acc - \frac{1}{2}\right) \sqrt[4]{\frac{y(\acc^2 y - \acc y + \acc - 1/2)}{\varrt}} \nonumber\\
\boundarysep &=& \frac{y}{\drift} \nonumber\\
\ndt &=& \meanrt - \left(\frac{\boundarysep}{2\drift}\right) \left(\frac{1-e^{-\drift\boundarysep}}{1+e^{-\drift\boundarysep}}\right) \nonumber
\end{eqnarray}
{where $\text{sign}(x)$ denotes the sign function, which returns $+1$ if $x \geq 0$ and $-1$ if $x < 0$. In this context, it determines whether drift is positive (when accuracy $\geq 0.5$) or negative (when accuracy $< 0.5$).}

These equations provide point estimates. The bootstrap method extends this to uncertainty quantification.

\subsection{The EZ bootstrap procedure applied to simple diffusion}

The bootstrap exploits the known sampling distributions of the summary statistics. For a sample of size $n$, these distributions are:
\begin{eqnarray*}
n\acc &\sim& \text{Binomial}(n, p)\\
\meanrt &\sim& \mathcal{N}\left(m, \frac{s^2}{n}\right)\\
\varrt &\sim& \text{Gamma}\left(\frac{n-1}{2}, \frac{2s^2}{n-1}\right)
\end{eqnarray*}
where $p$, $m$, and $s^2$ are the true population values of accuracy, mean RT, and RT variance.

\subsection{Distributional assumptions and their justification} The binomial distribution for accuracy is exact. The normal distribution for mean RT is asymptotically valid by the central limit theorem; for finite samples, this approximation is accurate when $n \geq 30$ and the underlying RT distribution has finite variance. The Gamma distribution for variance is exact under the assumption that RTs are normally distributed; while RTs are not exactly normal, this approximation is reasonable for the EZ model when sample sizes are moderate ($n \geq 50$).
Note, however, that we do not make sample size recommendations based on these classical rules of thumb.  The procedure we propose is so fast that Monte Carlo studies are very feasible methods to evaluate the effect of violated assumptions.

\subsection{Independence assumption} We assume approximate independence between the three summary statistics. While these statistics are derived from the same data, they are asymptotically independent under the EZ model assumptions. For finite samples, correlations have negligible impact on coverage properties (see our simulation studies below). 

\subsection{Simulation studies}

We present two standard calibration studies to validate the EZ bootstrap method for the EZ diffusion model.  In both cases, the target of the study was the calibration of the uncertainty quantification in terms of credibility interval coverage -- that is, we sought to confirm that credibility intervals contain the true value at a rate commensurate to their nominal coverage probability (specifically, that 95\% credible intervals contain the true value 95\% of the time).

\subsubsection{Study 1: Single condition}
In Study 1, we evaluated coverage for the regular EZ diffusion model with a single condition. We simulated $n = 100$ trials from true parameters $\boundarysep = 1.0$, $\drift = 0.5$, and $\ndt = 0.2$. For each of 10,000 simulation repetitions, we computed credible intervals at the 95\% confidence level using the EZ bootstrap method (with $1000$ bootstrap samples). 

% \textit{Results.}
We evaluated coverage at the 95\% confidence level across all simulation repetitions. The EZ bootstrap method achieved excellent calibration. Coverage rates were 95.5\%, 94.8\%, and 96.0\% for boundary, drift, and non-decision time, respectively---all within expected sampling variation of the nominal 95\% level. When considering all three parameters simultaneously, the coverage rate was 87.7\%, closely matching the expected value ($0.95^3 \approx 0.857$).

\subsubsection{Study 2: Design matrix extension}
In Study 2, we evaluated coverage for the EZ diffusion model with a design matrix extension. We simulated data from four conditions using a design matrix with three regression weights for each parameter type (boundary, drift, and non-decision time), resulting in 9 beta weight parameters total. The true beta weights were: boundary weights $[1.0, 1.5, 2.0]$, drift weights $[0.4, 0.8, 1.2]$, and non-decision time weights $[0.2, 0.3, 0.4]$. {These parameter values were chosen to represent typical values in the EZ diffusion model \cite{matzke2009psychological}.}

The design matrices for boundary, drift, and non-decision time parameters were:
$$
X_\alpha = \begin{pmatrix}
1 & 0 & 0 \\
0 & 1 & 0 \\
0 & 0 & 1 \\
0 & 0 & 1
\end{pmatrix}\!\!,\,\,\,
X_\delta = \begin{pmatrix}
1 & 0 & 0 \\
0 & 1 & 0 \\
0 & 0 & 1 \\
0 & 1 & 0
\end{pmatrix}\!\!,\,\,\,
X_\tau = \begin{pmatrix}
1 & 0 & 0 \\
0 & 1 & 0 \\
0 & 0 & 1 \\
1 & 0 & 0
\end{pmatrix} 
$$
where each row corresponds to a condition and each column corresponds to a regression weight.
Each condition had $n = 100$ trials. We conducted 10,000 simulation repetitions, computing credible intervals at the 95\% confidence level using EZ bootstrap (with $1000$ bootstrap samples).

The EZ bootstrap method maintained excellent calibration even with the added complexity of the design matrix structure. For boundary weights, coverage rates were 95.7\%, 94.9\%, and 96.6\% across the three weights. For drift weights, coverage rates were 94.6\%, 94.2\%, and 98.0\%. For non-decision time weights, coverage rates were 96.2\%, 94.3\%, and 94.7\%. All coverage rates were within expected sampling variation of the nominal 95\% level.

\subsubsection{Computational performance} 
The EZ bootstrap method is exceptionally computationally efficient. The simulations were run on an off-the-shelf laptop, using a single core of an 11th Gen Intel\textregistered{} Core\texttrademark{} i7-1195G7 processor running at 2.90GHz.  For Study~1, the method required a mean of only 0.16~ms per estimation (SD = 0.02~ms), enabling rapid uncertainty quantification. For Study~2, with the added complexity of the design matrix, the method required a mean of 0.95~ms per estimation (SD = 0.13~ms). A comparable Bayesian analysis reported by \citeauthor{chávez_de_la_peña_vandekerckhove:in_press:hierarchical} (\citeyearNP{chávez_de_la_peña_vandekerckhove:in_press:hierarchical}, Appendix~D) took approximately 3,800~ms. The sub-millisecond computation times of the EZ bootstrap demonstrate that the method makes real-time\footnote{We have a strict definition of ``real-time'' as less than the duration of one display frame in a typical experimental setup -- usually at most 1/120th of a second or 8.33~ms.} uncertainty quantification feasible where traditional methods would be prohibitively slow.


\section{Application to the circular diffusion model}

The circular diffusion model \cite{Smith2016} extends the diffusion model to circular decision spaces, where evidence accumulation occurs in a two-dimensional disk rather than along a one-dimensional line. The EZ version of the circular diffusion model (EZ-CDM; \citeNP{QarehdaghiRad2024}) provides analytical expressions that map four summary statistics---circular mean of choice angles, circular variance of choice angles, mean response time, and response time variance---to four cognitive parameters: drift angle, drift magnitude, boundary radius, and non-decision time. This makes the EZ-CDM framework a synthetic likelihood model: it provides a likelihood function based on summary statistics rather than raw trial data.

\subsection{The EZ circular diffusion model}

The EZ circular diffusion model provides analytical expressions for four summary statistics from a set of $n$ circular decision trials.  Let $\MCA{}$ denote the observed circular mean of choice angles, $\VCA{}$ the observed circular variance of choice angles, $\MRT{}$ the observed mean response time, and $\VRT{}$ the observed response time variance. The model's four parameters are the drift angle $\driftangle$, drift magnitude $\driftmag$, boundary radius $\boundaryradius$, and non-decision time $\ndtcdm$.
This allows us to define forward equations 
$
f_\text{CDM}:
\left(\driftangle, \driftmag, \boundaryradius, \ndtcdm\right)
\mapsto
\left(\MCA{}, \VCA{}, \MRT{}, \VRT{}\right)
$ and inverse equations 
$
i_\text{CDM}:
\left(\MCA{}, \VCA{}, \MRT{}, \VRT{}\right)
\mapsto
\left(\driftangle, \driftmag, \boundaryradius, \ndtcdm\right)
$.

The forward equations $f_\text{CDM}: \left(\driftangle, \driftmag, \boundaryradius, \ndtcdm\right) \mapsto \left(\MCA{}, \VCA{}, \MRT{}, \VRT{}\right)$ map parameters to summary statistics. The choice-angle distribution is von Mises with mean $\mu = \driftangle$ and concentration $\kappa = \boundaryradius\driftmag$.
The circular mean and variance are then:
\begin{eqnarray}
\MCA{} &=& \driftangle \nonumber\\
\VCA{} &=& 1 - R \nonumber
\end{eqnarray}
where $R = {I_1(\kappa)}/{I_0(\kappa)}$ and $I_0$ and $I_1$ are modified Bessel functions of the first kind. The response time moments are:
\begin{eqnarray}
\MRT{} &=& \ndtcdm + \frac{\boundaryradius}{\driftmag} R \nonumber\\
\VRT{} &=& \frac{\boundaryradius^2}{\driftmag^2} R^2 + \frac{2\boundaryradius}{\driftmag^3} R - \boundaryradius^2 \nonumber
\end{eqnarray}

The inverse equations $i_\text{CDM}: \left(\MCA{}, \VCA{}, \MRT{}, \VRT{}\right) \mapsto \left(\driftangle, \driftmag, \boundaryradius, \ndtcdm\right)$ map summary statistics to parameters. The estimation proceeds in four steps:
\begin{eqnarray}
\hat{\driftangle} &=& \MCA{} \nonumber\\
R &=& 1 - \VCA{}, \quad \text{solve } \frac{I_1(\kappa)}{I_0(\kappa)} = R \text{ for } \kappa \nonumber\\
\hat{\driftmag}^2 &=& \frac{-\kappa^2 + \sqrt{\kappa^4 + 4\VRT{}(\kappa^2 R^2 + 2\kappa R)}}{2\VRT{}}, \quad \hat{\boundaryradius} = \frac{\kappa}{\hat{\driftmag}} \nonumber\\
\hat{\ndtcdm} &=& \MRT{} - \frac{\hat{\boundaryradius}}{\hat{\driftmag}}R \nonumber
\end{eqnarray}
where the solution for $\hat{\driftmag}^2$ is obtained by solving the quadratic equation derived from the forward equation for $\VRT{}$.
Solving $R = I_1(k)/I_0(k)$ for $k$ is faciliated by the piecewise estimator of \citeA[see pp.~88-93 for further detail]{Fisher1993}: $$k \approx \begin{cases} 2R + R^{3} + \tfrac{5}{6}R^{5}, & R < 0.53,\\ -0.4 + 1.39R + \dfrac{0.43}{1-R}, & 0.53 \le R < 0.85,\\ \dfrac{1}{R^{3} - 4R^{2} + 3R}, & R \ge 0.85.\end{cases}$$
As an excellent alternative, \citeA{QarehdaghiRad2024} suggest performing a single step of Newton-Raphson optimization. This approach typically provides higher accuracy than the piecewise approximation but requires extra computational effort.

% \textit{Sampling distributions.}
The sampling distributions of the four summary statistics determine the variability and robustness of EZ-CDM parameter estimates. Under the CDM, choice angles follow a von Mises distribution with concentration $\kappa = \boundaryradius\driftmag$, and response times follow the first-passage-time distribution of a two-dimensional Bessel process with drift.

For the circular mean of choice angles (MCA), the angular error $\widehat{\driftangle} - \driftangle$ is approximately normally distributed for large $N$ \cite{Fisher1993}, and so:
\begin{equation}
\widehat{\driftangle} \sim \mathcal{N}\left(\driftangle, \frac{1}{N\kappa R}\right). \nonumber
\end{equation}
The variance decreases rapidly with concentration $\kappa$, making estimation more difficult when drift magnitude or boundary radius are small.

For the circular variance ($\VCA{}$), the mean resultant length $L = \sqrt{\left(\frac{1}{N}\sum_{i=1}^{N} \cos(\theta_i)\right)^2 + \left(\frac{1}{N}\sum_{i=1}^{N} \sin(\theta_i)\right)^2}$ is approximately normal for moderate-to-large $N$, with mean $R$ and variance $\frac{1 - R^2}{N}$ \cite{Fisher1993}. Consequently, $\widehat{\VCA{}} = 1 - R$ is also asymptotically normal with the same variance. The circular variance is only weakly sensitive to outliers in angle (since angles wrap on the circle). However, parameter estimation becomes unstable when $R$ is near zero (because then the data are essentially uniform on the circle and provide limited information about the concentration parameter $\kappa$).

For mean response time ($\MRT{}$), the central limit theorem applies:
\begin{equation}
\widehat{\MRT{}} \sim \mathcal{N}\left(\MRT{}, \frac{\VRT{}}{N}\right) \nonumber
\end{equation}
Mean RT is robust for large $N$ but sensitive to RT contaminants.

For response time variance ($\VRT{}$), the sample variance is asymptotically normal but with slower convergence:
\begin{equation}
\widehat{\VRT{}} \sim \mathcal{N}\left(\VRT{}, \frac{m_4 - \VRT{}^2}{N}\right) \nonumber 
\end{equation}
where $m_4$ is the fourth central moment of the RT distribution. This statistic is the most contamination-sensitive in EZ-CDM, as a small number of extreme RTs can cause large upward shifts (see \citeNP{chávez_de_la_peña_etal:preprint:hierarchical}, and \citeNP{QarehdaghiRad2024}, for approaches using robust summary statistics).

The EZ bootstrap procedure can be applied to EZ-CDM by resampling these statistics from their asymptotic distributions. The angle-based statistics ($\MCA{}$ and $\VCA{}$) have well-behaved sampling distributions, while the RT-based statistics ($\MRT{}$ and $\VRT{}$) introduce most of the sampling variability in the inversion system, especially when solving for drift magnitude and boundary radius.

\subsection{Simulation studies}

We again present two calibration studies to validate the EZ bootstrap method for the EZ circular diffusion model.  The target of the study was again the calibration of the uncertainty quantification in terms of credibility interval coverage (95\% credibility).

\subsubsection{Study 1: Single condition}
In Study 1, we evaluated coverage for the regular EZ circular diffusion model with a single condition. We simulated $n = 100$ trials from true parameters $\driftangle = 0.0$, $\driftmag = 1.0$, $\boundaryradius = 1.0$, and $\ndtcdm = 0.2$. For each of 10,000 simulation repetitions, we computed credible intervals at the 95\% confidence level using the EZ bootstrap method (with $1000$ bootstrap samples). 

% \textit{Results.}
We evaluated coverage at the 95\% confidence level across 10,000 simulation repetitions. The EZ bootstrap method achieved excellent calibration. Coverage rates were 93.9\%, 94.3\%, 95.2\%, and 94.8\% for drift angle, drift magnitude, boundary radius, and non-decision time, respectively---all within expected sampling variation of the nominal 95\% level. When considering all four parameters simultaneously, the coverage rate was 82.4\%, closely matching the expected value ($0.95^4 \approx 0.815$).

\subsubsection{Study 2: Design matrix extension}
In Study 2, we evaluated coverage for the EZ circular diffusion model with a design matrix extension. We simulated data from four conditions using a design matrix with three regression weights for each parameter type (drift angle, drift magnitude, boundary radius, and non-decision time), resulting in 12 beta weight parameters total. The true beta weights were: drift angle weights $[0.0, 0.5, 1.0]$, drift magnitude weights $[0.5, 0.7, 0.9]$, boundary radius weights $[0.8, 1.0, 1.2]$, and non-decision time weights $[0.15, 0.2, 0.25]$ \cite{villarreal_etal:2024:graphical}.

The design matrices for drift angle, drift magnitude, boundary radius, and non-decision time parameters were:
$$
X_{\driftangle} = \begin{pmatrix}
1 & 0 & 0 \\
0 & 1 & 0 \\
0 & 0 & 1 \\
0 & 1 & 0
\end{pmatrix}\!\!,\,\,\,
X_{\driftmag} = \begin{pmatrix}
1 & 0 & 0 \\
0 & 1 & 0 \\
0 & 0 & 1 \\
0 & 1 & 0
\end{pmatrix}\!\!,$$
$$X_{\boundaryradius} = \begin{pmatrix}
1 & 0 & 0 \\
0 & 1 & 0 \\
0 & 0 & 1 \\
1 & 0 & 0
\end{pmatrix}\!\!,\,\,\,
X_{\ndtcdm} = \begin{pmatrix}
1 & 0 & 0 \\
0 & 1 & 0 \\
0 & 0 & 1 \\
0 & 0 & 1
\end{pmatrix} 
$$
where each row corresponds to a condition and each column corresponds to a regression weight.
Each condition had $n = 100$ trials. We conducted 10,000 simulation repetitions, computing credible intervals at the 95\% confidence level using EZ bootstrap (with $1000$ bootstrap samples).

The method maintained excellent calibration: For drift angle weights, coverage rates were 89.5\%, 94.5\%, and 94.2\% across the three weights. For drift magnitude weights, coverage rates were 94.5\%, 94.4\%, and 89.3\%. For boundary radius weights, coverage rates were 95.8\%, 94.8\%, and 93.0\%. For non-decision time weights, coverage rates were 95.8\%, 94.6\%, and 95.7\%. All coverage rates were within expected sampling variation of the nominal 95\% level.

\subsubsection{Computational performance}
For Study~1, the method required a mean of only 0.33~ms per estimation (SD = 0.03~ms), enabling rapid uncertainty quantification. For Study~2, with the added complexity of the design matrix, the method required a mean of 1.26~ms per estimation (SD = 0.19~ms). 

\section{Other models}

The EZ bootstrap method generalizes to any model in the EZ class -- that is, any model with analytical forward and inverse equations smoothly mapping parameters to summary statistics. Here we briefly describe one additional example: the one-high-threshold source-monitoring {multinomial processing tree (MPT)} model.

\subsection{One-high-threshold source-monitoring model}

The one-high-threshold source-monitoring MPT model \cite{BatchelderRiefer1990,BatchelderRiefer1999} is a multinomial processing tree model for source memory that allows participants to identify items as coming from Source~A, Source~B, or as new. The model has four parameters: detection probability $\Detect$, source discrimination probability $\Discrim$, criterion for new items $\Criterion$, and guessing probability $\Guess$.

The forward equations map parameters to observed response proportions. For old items from Source~A, the forward equations are:
\begin{eqnarray}
\AAstat &=& \Detect\Discrim + (1-\Detect)\Guess \nonumber\\
\BAstat &=& \Detect(1-\Discrim) + (1-\Detect)(1-\Guess) \nonumber
\end{eqnarray}
where $\AAstat = P(\text{``A''} \mid \text{old from A})$ and $\BAstat = P(\text{``B''} \mid \text{old from A})$. Analogous expressions hold for old items from Source~B:
\begin{eqnarray}
\ABstat &=& \Detect(1-\Discrim) + (1-\Detect)\Guess \nonumber\\
\BBstat &=& \Detect\Discrim + (1-\Detect)(1-\Guess) \nonumber
\end{eqnarray}
where $\ABstat = P(\text{``A''} \mid \text{old from B})$ and $\BBstat = P(\text{``B''} \mid \text{old from B})$. For new items, the model specifies:
\begin{eqnarray}
\NNstat &=& \Criterion \nonumber\\
\ANstat &=& (1-\Criterion)\Guess \nonumber\\
\BNstat &=& (1-\Criterion)(1-\Guess) \nonumber
\end{eqnarray}
where $\NNstat = P(\text{``new''} \mid \text{new})$, $\ANstat = P(\text{``A''} \mid \text{new})$, and $\BNstat = P(\text{``B''} \mid \text{new})$.
The inverse equations provide closed-form parameter estimates:
\begin{eqnarray}
\Criterion &=& \NNstat \nonumber\\
\Guess &=& \frac{\ANstat}{1-\NNstat} \nonumber\\
\Detect &=& \frac{\AAstat + \ABstat - 2\Guess}{1-2\Guess} \nonumber\\
\Discrim &=& \frac{\AAstat - (1-\Detect)\Guess}{\Detect} \nonumber
\end{eqnarray}
These linear relationships allow closed-form inversion, illustrating a simple but nontrivial MPT for which both forward and inverse transformations are available in closed form.

The sampling distributions for MPT models are multinomial: the observed counts in each response category follow a multinomial distribution with probabilities given by the forward equations. For old items from Source~A, the counts $(n_{\AAstat}, n_{\BAstat})$ follow:
\begin{equation}
(n_{\AAstat}, n_{\BAstat}) \sim \text{Multinomial}(n_A, (\AAstat, \BAstat)) \nonumber
\end{equation}
where $n_A$ is the number of old items from Source~A and $\AAstat + \BAstat = 1$. Similarly, for old items from Source~B:
\begin{equation}
(n_{\ABstat}, n_{\BBstat}) \sim \text{Multinomial}(n_B, (\ABstat, \BBstat)) \nonumber
\end{equation}
where $n_B$ is the number of old items from Source~B and $\ABstat + \BBstat = 1$. For new items:
\begin{equation}
(n_{\NNstat}, n_{\ANstat}, n_{\BNstat}) \sim \text{Multinomial}(n_N, (\NNstat, \ANstat, \BNstat)) \nonumber
\end{equation}
where $n_N$ is the number of new-item trials and $\NNstat + \ANstat + \BNstat = 1$. The EZ bootstrap procedure can be applied by resampling from these multinomial distributions and transforming through the inverse equations to obtain bootstrap parameter estimates.

We provide this solvable MPT example only as an illustration. The class of MPT models with tractable analytical inverses is rather narrow and does not reflect the complexity of most MPT applications, where estimation usually requires numerical optimization or Bayesian inference \cite{matzke_etal:2018:nonstandard}.

\section{Discussion}

The EZ bootstrap method provides fast uncertainty quantification. It maintains proper calibration while achieving dramatic speed improvements over previous methods such as the Bayesian implementation by \citeA{chávez_de_la_peña_vandekerckhove:in_press:hierarchical} -- the method achieves coverage rates that closely match nominal levels for both simple single-condition models and more complex design matrix extensions.

The method's success relies on the synthetic likelihood structure of EZ models. By operating directly on summary statistics with known sampling distributions, we avoid the computational cost of raw data resampling while preserving statistical validity. The transformation-of-variables approach generalizes to any model in the EZ class -- that is, any model with invertible mappings from parameters to summary statistics. The calibration results illustrate that our method maintains proper statistical properties while delivering computational efficiency.


\subsection{Limitations}

{The bootstrap method has a few limitations worth pointing out. First, the method is limited to the class of EZ models and their design matrix extensions. While identifiable models will typically \textit{have} both the forward and inverse equations, the equations are often not known or are mathematically inconvenient. Second, the method requires that summary statistics have known or approximately known sampling distributions. Approximations such as the normal distribution for mean RT and the Gamma for variance are asymptotically valid but may be less accurate for very small samples ($n < 30$) or when RT distributions are highly skewed, but good approximations of sampling distributions are known for very many summary statistics. Third, we assume approximate independence between summary statistics. While this holds asymptotically and is approximately valid for sufficiently large samples, correlations may be non-negligible for very small samples or models beyond our consideration.}


\subsection{Future directions}

Several extensions are possible. The method could be extended to yet more models with known mappings from parameters to summary statistics. Hierarchical extensions could enable multi-level modeling with summary statistics. Integration with experimental platforms could enable real-time semi-Bayesian design optimization and computerized adaptive tests.

Perhaps most promisingly, implicit mappings from parameters to summary statistics exist even when explicit equations are unknown. Techniques from statistical learning, such as normalizing flow neural networks, could be used to approximate these mappings for use in the synthetic bootstrap method, extending the approach to a broader class of models.


\section{Computational reproducibility}

All code associated with this project is available via GitHub. The simulation studies reported in this paper can be reproduced using the C implementation available in the GitHub repository \texttt{joachimvandekerckhove/ez-bootstrap}. The implementation requires the GNU Scientific Library (GSL) for random number generation and statistical distributions.

To reproduce the simulation studies, first
(1)~clone the repository, then
(2)~install dependencies if needed (e.g., install \texttt{libgsl-dev} to obtain GSL development libraries on Debian/Ubuntu systems), and 
(3)~run the simulation script
\texttt{./run\_all.sh}.

The script executes both simulation studies for the EZ-DDM and EZ-CDM with the following editable parameters: 10,000 simulation repetitions, sample size $n = 100$ trials per condition, and 1,000 bootstrap samples per repetition. The simulation outputs include coverage rates for each parameter (or beta weight) and computational performance metrics.

{A Python implementation of the EZ-DDM is also available as part of the larger EZAS package (\url{https://github.com/joachimvandekerckhove/ezas}), which uses the same method but is not as fast as the C implementation.}


\ifdefined\apaformat

\newcommand{\dec}[2]{\noindent\textbf{#1} --- #2}

\section{Declarations}

\dec{Funding}{This work was supported by the National Research Council.}
   
\dec{Competing interests}{The authors report no competing interest.}

\dec{Ethics approval}{N/A}

\dec{Consent to participate}{N/A}

\dec{Consent for publication}{N/A}

\dec{Availability of data and materials}{N/A}

\dec{Code availability}{See Section ``Computational reproducibility.''}

\fi


\bibliography{includes/references,includes/cidlab}\null

\subsection{Disclaimer}
The views expressed are those of the authors and do not necessarily reflect the official policy or position of the Department of the Air Force, the Department of Defense, or the U.S. Government. 


% \appendix
% \ifdefined\apaformat
% \section{Reproducing the simulation studies}
% \else
% \section{Appendix: Reproducing the simulation studies}
% \fi

\end{document}
"
% To compile in original format: pdflatex main.tex

% \def\apaformat{1}

\ifdefined\apaformat
  % APA-style preamble for draft format
% This file is included by main.tex when compiling in APA format

\documentclass[man,12pt]{apa6}

\usepackage{
  amsmath,amssymb,amsfonts,
  csquotes,
  epsfig,epstopdf,epigraph,
  fancyhdr,
  graphicx,
  hyperref,
  mathtools,
  setspace,
  subfig,
  tikz,
  url,
  apacite}

% Set up bibliography style for APA format
\bibliographystyle{apacite}

% Redefine cidlab-specific commands to work with apa6
\newcommand{\leadauthor}[1]{}
\newcommand{\significancestatement}[1]{}
\newcommand{\authorcontributions}[1]{}
\newcommand{\authordeclaration}[1]{}
\newcommand{\equalauthors}[1]{}
\newcommand{\correspondingauthor}[1]{}
\newcommand{\dates}[1]{}
\newcommand{\verticaladjustment}[1]{}
\newcommand{\firststyle}{fancy}
\newcommand{\abscontentformatted}{}
\newcommand{\abscontent}{}
\newcommand{\nocaption}{}

% Redefine section commands for APA format
\let\origsection\section
\let\origsubsection\subsection
\renewcommand{\section}[1]{\origsection{#1}}
\renewcommand{\subsection}[1]{\origsubsection{#1}}

% Redefine citation commands for APA format
% \citeNP is already defined by apacite

% Define introduction command for APA format
\newcommand{\introduction}[1]{\section{Introduction}\label{sec:introduction}}


% Include shared metadata
\input{metadata}

\else
  % Original cidlab-style preamble for pretty format
% This file is included by main.tex when compiling in original format

\documentclass[9pt,twocolumn,twoside]{cidlab/cidlab}

\usepackage{
  amsmath,amssymb,amsfonts,
  csquotes,
  epsfig,epstopdf,epigraph,
  graphicx,
  mathtools,
  subfig,
  tikz,
  url}
\usepackage[super]{nth}
\usepackage[english]{babel}

% Define introduction command that does nothing in cidlab format
\newcommand{\introduction}[1]{}


% Abstract for cidlab format
\begin{abstract} 
\input{includes/abstract}
\end{abstract}

\significancestatement{}
\authorcontributions{}
\authordeclaration{}
\equalauthors{}
\correspondingauthor{\textsuperscript{$\dagger$}To whom correspondence should be addressed. E-mail: joachim@uci.edu.}

\dates{Draft of \today.}
% \dates{Planned outlet: \textit{Nature Human Behaviour}.}

% Include shared metadata
\input{metadata}

\AtBeginDocument{
\verticaladjustment{-2pt}
\thispagestyle{firststyle}
\ifthenelse{\boolean{shortarticle}}{\ifthenelse{\boolean{singlecolumn}}{\abscontentformatted}{\abscontent}}{}
}

\fi
% Main document content starts here


\setlength {\marginparwidth }{2cm}


\usepackage{todonotes}

\newif\ifshownotes

\newcommand{\definenote}[2]{%
  \expandafter\newcommand\csname #1\endcsname[1]{%
    \ifshownotes
      \todo[inline,color=#2]{##1 -\textit{#1}}%
    \fi
  }%
}

\shownotesfalse

% https://en.wikipedia.org/wiki/X11_color_names
\definenote{jv}{Lime}
\definenote{elf}{Gold}


% Comment this out to remove notes:
\shownotestrue

% Macros for EZ model summary statistics and parameters
% This file contains all custom macros used throughout the document

% Signal Detection Theory (SDT) parameters
\newcommand{\dprime}{d'}
\newcommand{\criterion}{c}

% Signal Detection Theory (SDT) summary statistics
\newcommand{\HitRate}{H}
\newcommand{\FalseAlarm}{F}

% EZ diffusion model parameters
\newcommand{\drift}{\delta}
\newcommand{\boundarysep}{\alpha}
\newcommand{\ndt}{\tau}

% EZ diffusion model summary statistics
\newcommand{\acc}{\hat{P}}
\newcommand{\meanrt}{\hat{M}}
\newcommand{\varrt}{\hat{V}}

% EZ circular diffusion model (EZ-CDM) parameters
\newcommand{\driftangle}{\theta_v}
\newcommand{\driftmag}{v}
\newcommand{\boundaryradius}{a}
\newcommand{\ndtcdm}{t_0}

% EZ circular diffusion model (EZ-CDM) summary statistics
\newcommand{\MCA}{M_a}
\newcommand{\VCA}{V_a}
\newcommand{\MRT}{M_t}
\newcommand{\VRT}{V_t}

% One-high-threshold source-monitoring MPT model parameters
\newcommand{\Detect}{D}
\newcommand{\Discrim}{d}
\newcommand{\Criterion}{c}
\newcommand{\Guess}{g}

% One-high-threshold source-monitoring MPT model summary statistics
\newcommand{\AAstat}{A_A}
\newcommand{\BAstat}{B_A}
\newcommand{\ABstat}{A_B}
\newcommand{\BBstat}{B_B}
\newcommand{\NNstat}{N_N}
\newcommand{\ANstat}{A_N}
\newcommand{\BNstat}{B_N}


\usepackage{algorithm}
\usepackage{algpseudocode}
\usepackage{listings}
\lstset{extendedchars=true,upquote=true}
% Handle UTF-8 characters in listings using literate option
\lstdefinestyle{utf8style}{
  extendedchars=true,
  upquote=true,
  literate={≈}{{$\approx$}}1
}
\lstset{style=utf8style}

\renewcommand{\subsubsection}[1]{\textit{#1}.}

\begin{document}

\maketitle

\introduction{}

\noindent{}%
Cognitive models serve dual purposes in psychological research. On one hand, they can be used to understand the cognitive infrastructure---the underlying mechanisms and processes that give rise to behavior. On the other hand, cognitive models can function as measurement tools, translating behavioral observations into interpretable psychological parameters. This paper focuses on the latter: we develop computationally efficient methods for uncertainty quantification in cognitive models used as measurement instruments. The need for computational efficiency becomes critical in large-scale applications such as meta-analyses across hundreds of studies, real-time experimental analysis during data collection, and individual differences research requiring parameter estimation for thousands of participants. Traditional approaches to uncertainty quantification, which rely on computationally expensive resampling or Bayesian inference, create barriers in these contexts. We propose a bootstrap method that exploits the synthetic likelihood structure of a class of cognitive models we call `EZ' models, achieving dramatic speed improvements while maintaining statistical validity.

\section{Synthetic likelihood methods}

Synthetic likelihood methods provide a framework for statistical inference when the likelihood function is intractable but summary statistics with known or estimable distributions are available \cite{wood_statistical_2010,Price2017}. The core idea is to replace the full likelihood with a likelihood based on summary statistics, which can be computed efficiently even when the full data likelihood cannot.

The synthetic likelihood approach was introduced by \citeauthor{wood_statistical_2010} for inference in complex ecological systems with chaotic dynamics. When direct likelihood evaluation is impossible, one can extract phase-insensitive summary statistics from the data and compare them to summary statistics simulated from the model. This approach has been extended to Bayesian settings \cite{Price2017}, where the synthetic likelihood is used within a Bayesian framework to enable posterior inference.

Certain cognitive models, which we will refer to here as the `EZ' models (after the EZ diffusion model; \citeNP{WagenmakersEtAl2007}), similarly fit the description of synthetic likelihood models, and provide a natural application of methods specific to that class. For example, in the EZ diffusion model, rather than working with the full likelihood over all trial-level data, we work with three summary statistics: accuracy rate, mean response time, and response time variance. These statistics have known sampling distributions, allowing us to construct a likelihood function directly from the summaries \cite{chávez_de_la_peña_vandekerckhove:in_press:hierarchical}. This synthetic likelihood approach is particularly powerful because the EZ model provides analytical expressions that map parameters to these summary statistics, eliminating the need for simulation, and paving the way for uncertainty quantification using a variation on a classic transformation-of-variables technique, which we will describe after introducing the model class.

\section{The class of EZ models}

We will rely on a class of cognitive models we call `EZ'.  EZ models are defined by the existence of an invertible system of equations that maps a model parameter vector $\theta$ to a set of summary statistics $S$.  The vector $\theta$ need not contain \textit{all} of the model's parameters, but may be limited to a subset of interest (as in the application by \citeNP{wood_statistical_2010}).  The system consists of a set of smooth `forward' equations $f$ and `inverse' equations $i$:
\begin{eqnarray}
    f& :&  \theta \mapsto S \nonumber\\
    i& :&  S \mapsto \theta\label{eq:generic:inv}
\end{eqnarray}

The EZ class contains some well-known cognitive models.  We take signal detection theory (SDT; \citeNP{swets_green:1966:signal}) as an example:
\begin{eqnarray}
    (\HitRate,\FalseAlarm) &=& f_\text{SDT}(\dprime,\criterion) \nonumber \\
    &\Leftrightarrow& \left\{
    \begin{array}{rcl}
         \HitRate&=& \Phi(\dprime - \criterion) \\
         \FalseAlarm&=& \Phi(-\criterion) \\
    \end{array}\right.
\end{eqnarray}
where $\Phi$ denotes the standard normal cumulative distribution function. The inverse equations map summary statistics back to parameters:
\begin{eqnarray}
    (\dprime,\criterion) &=& i_\text{SDT}(\HitRate,\FalseAlarm) \nonumber \\
    &\Leftrightarrow& \left\{
    \begin{array}{rcl}
         \dprime&=& \Phi^{-1}(\HitRate) - \Phi^{-1}(\FalseAlarm) \\
         \criterion&=& -\Phi^{-1}(\FalseAlarm) \\
    \end{array}\right.
\end{eqnarray}
where $\Phi^{-1}$ denotes the inverse standard normal cumulative distribution function. Here, $\HitRate$ denotes the hit rate (probability of correctly identifying a signal), $\FalseAlarm$ denotes the false alarm rate (probability of incorrectly identifying a signal when none is present), $\dprime$ denotes sensitivity (the ability to discriminate signal from noise), and $\criterion$ denotes the criterion (the decision threshold).

Other members of the EZ class include the EZ diffusion model (\citeNP{WagenmakersEtAl2007}), the EZ circular diffusion model \cite{QarehdaghiRad2024}, and some simple multinomial processing tree models with analytical solutions. {These models are discussed in detail below.}


\section{The EZ bootstrap procedure}\label{sec:ez_bootstrap}

We propose a bootstrap method we call `EZ bootstrap' that exploits the synthetic likelihood structure of EZ models. The method operates directly on summary statistics, using their known sampling distributions to perform a parametric bootstrap. 

In the abstract, EZ bootstrap proceeds along the following steps. First, (1) compute summary statistics $S$ from the {observed} data. Then, (2) for each bootstrap repetition $b = 1, \ldots, B$: (2a) sample summary statistics $S^*_b$ from their known sampling distributions $\mathcal{D}_S$ (the requirement that these distributions be known or estimable is fundamental to the method), and (2b) transform $S^*_b$ through the inverse equations $i$ to obtain bootstrap parameter estimates $\theta^*_b = i(S^*_b)$. Finally, (3) aggregate the bootstrap parameter estimates $\{\theta^*_1, \ldots, \theta^*_B\}$ to compute means, standard deviations, and credible intervals.

This procedure is formalized in Algorithm~\ref{alg:ez_bootstrap}. EZ bootstrap is fundamentally a transformation-of-variables approach to uncertainty estimation. Rather than resampling raw data and re-estimating parameters, we resample the summary statistics from their known distributions, then transform these resampled statistics through the inverse equations to obtain parameter estimates. {Specifically, in Algorithm~\ref{alg:ez_bootstrap} we sample $S^*_b$ from $\mathcal{D}_S$ for each bootstrap repetition $b = 1, \ldots, B$, then compute $\theta^*_b = i(S^*_b)$ to obtain the bootstrap parameter estimates $\{\theta^*_1, \ldots, \theta^*_B\}$.} This transformation preserves the statistical properties of the bootstrap while eliminating the computational cost of data resampling. 

\subsection{Transformation of variables for uncertainty quantification}
Transformation of variables is a fundamental method in statistics for propagating uncertainty through deterministic transformations (\citeNP{rice2006mathematical}, see also \citeNP{hayes2013relative}). When a random variable $X$ with known distribution is transformed through a function $Y = g(X)$, the distribution of $Y$ can be derived using the change-of-variables formula. This technique is central to the EZ bootstrap method.

In the context of uncertainty quantification, transformation of variables allows us to derive the distribution of parameter estimates from the distribution of summary statistics in EZ models. The inverse equations (Eq.~\ref{eq:generic:inv}) provide a deterministic transformation from summary statistics to parameters. By applying this transformation to bootstrap samples of the summary statistics, we obtain bootstrap samples of the parameters, which can then be used to quantify uncertainty.

The key advantage of this approach is computational efficiency. Rather than resampling raw data and re-estimating parameters for each bootstrap sample---which would require expensive optimization or simulation \cite{efron1994introduction}---we resample the summary statistics from their known distributions and apply the analytical transformation, {as formalized in Algorithm~\ref{alg:ez_bootstrap}}. This eliminates the computational bottleneck while preserving the statistical properties of the bootstrap.

The transformation-of-variables approach is made possible in EZ models particularly because the inverse transformation is analytical, smooth, and one-to-one (within the parameter space, or a relevant subspace thereof). 

\begin{algorithm}[bt]
\caption{EZ Bootstrap Procedure}
\label{alg:ez_bootstrap}
\begin{algorithmic}[1]
\Require Observed summary statistics $S$, number of bootstrap samples $B$, inverse transformation $i: S \mapsto \theta$, sampling distributions $\mathcal{D}_S$ for each component of $S$
\Ensure Bootstrap parameter estimates $\{\theta^*_1, \ldots, \theta^*_B\}$
\State $\Theta^* \leftarrow \emptyset$
\For{$b = 1$ to $B$}
    \State Sample $S^*_b \sim \mathcal{D}_S$ from known sampling distributions of $S$
    \State $\theta^*_b \leftarrow i(S^*_b)$
    \State $\Theta^* \leftarrow \Theta^* \cup \{\theta^*_b\}$
\EndFor
\Return $\Theta^*$
\end{algorithmic}
\end{algorithm}

\subsection{Theoretical justification for coverage preservation} The bootstrap procedure maintains proper coverage properties when the inverse transformation $i: S \mapsto \theta$
is a continuous, differentiable function. Under the bootstrap sampling scheme, we generate $S^*$ from its known distribution, then compute $\theta^* = i(S^*)$. Since $i$ is deterministic and the bootstrap samples $S^*$ have the correct sampling distributions, the transformed samples $\theta^*$ correctly represent the sampling distribution of the parameter estimates. This follows from the invariance property of bootstrap methods under smooth transformations \cite{rice2006mathematical}. The coverage properties are preserved because the bootstrap distribution of the transformed parameters matches the true sampling distribution of the parameter estimates -- up to the accuracy of the assumed distributions for the summary statistics.

\section{Design matrix extension}

In applied contexts, we often constrain parameters across conditions. These constraints can be expressed using design matrices. For example, {for diffusion models,} if drift rates, {$\drift$,} vary linearly with a stimulus property $x_i$, we can write:
$$\drift_i = \beta_0 + \beta_1 x_i.$$
If they vary quarilinearly, then:
$$\drift_i = \beta_0 + \beta_1 x_i + \beta_2 x_i^2.$$

This generalizes, so that parameter constraints can be expressed as:
$$\Delta = X \times \boldsymbol{\beta}$$
where $\Delta$ is a vector of parameters, $X$ is a design matrix, and $\boldsymbol{\beta}$ is a vector of regression weights.

If the model parameters $\Delta$ are known and the model is identified, the weights $\boldsymbol{\beta}$ can be derived via matrix inversion:
\begin{equation}
    \hat{\boldsymbol{\beta}} = X^{-1} \times \Delta \label{eq:design:solution}
\end{equation}

Importantly, adding the design matrix extension \textit{keeps the model inside the EZ class} and does not change the bootstrap logic.  This is easy to prove: Let $i: S \mapsto \theta$ denote the EZ inverse transformation mapping summary statistics $S$ to parameters $\theta$ (Eq.~\ref{eq:generic:inv}), and let $h: \theta \mapsto \boldsymbol{\beta}$ denote the design matrix transformation mapping parameters $\theta$ to beta weights $\boldsymbol{\beta}$ via $h(\theta) = X^{-1}\theta$ (where $X$ is the design matrix). The basic EZ bootstrap generates bootstrap samples $\theta^* = i(S^*)$ where $S^*$ is resampled from the known distribution of summary statistics. With the design matrix extension, we generate bootstrap samples $\boldsymbol{\beta}^* = h(i(S^*)) = X^{-1}i(S^*)$. 
Since $i$ is a deterministic, smooth transformation (by the definition of EZ models) and $h$ is a deterministic linear transformation (matrix multiplication), their composition $\left(h \circ i\right): S \mapsto \boldsymbol{\beta}$ is also deterministic and smooth. By the same invariance property of bootstrap methods under smooth transformations \cite{rice2006mathematical}, if the bootstrap distribution of $S^*$ correctly represents the sampling distribution of $S$, then the bootstrap distribution of $\boldsymbol{\beta}^* = (h \circ i)(S^*)$ correctly represents the sampling distribution of $\boldsymbol{\beta} = (h \circ i)(S)$. Therefore, the design matrix extension preserves the bootstrap's coverage properties: the model remains in the EZ class because we still use the EZ inverse transformation, and the bootstrap logic is unchanged because we are still propagating uncertainty through a deterministic transformation.

\section{Application to the simple diffusion model}

The drift diffusion model (DDM) provides a unified account of decision-making processes, modeling both response accuracy and response time as the result of an evidence accumulation process \cite{Ratcliff1978,Wagenmakers2009Bertelson}. This framework has proven valuable in cognitive psychometrics, enabling the translation of behavioral patterns into interpretable cognitive parameters \cite<e.g.,>{pe_etal:2013:relationship,schubert_etal:2019:neuroscience,vandekerckhove:2014:simultaneous}.

The EZ diffusion model \cite{WagenmakersEtAl2007} was originally proposed as a method to simplify use of the DDM by providing analytical expressions that map three summary statistics---accuracy rate, mean response time, and response time variance---to three cognitive parameters: drift rate, boundary separation, and non-decision time. This makes the EZ framework a synthetic likelihood model: it provides a likelihood function based on summary statistics rather than raw trial data.

\subsection{The EZ diffusion model}

The EZ diffusion model provides analytical expressions for three summary statistics from a set of $n$ two-choice decision trials.  Let $\acc$ denote the observed accuracy rate, $\meanrt$ the observed mean response time, and $\varrt$ the observed response time variance. The model's three parameters are the drift rate $\drift$, boundary separation $\boundarysep$, and non-decision time $\ndt$.
This allows us to define forward equations 
$
f_\text{DIF}:
\left(\drift,\boundarysep,\ndt\right)
\mapsto
\left(\acc, \meanrt, \varrt\right)
$ and inverse equations 
$
i_\text{DIF}:
\left(\acc, \meanrt, \varrt\right)
\mapsto
\left(\drift,\boundarysep,\ndt\right)
$.

The EZ forward equations map parameters to summary statistics. Let $y = \exp(-\boundarysep \drift)$ where $\boundarysep$ is boundary separation and $\drift$ is drift rate. Then,
\begin{eqnarray}
\acc &=& \frac{1}{1+y} \nonumber\\
\meanrt &=& \ndt + \left(\frac{\boundarysep}{2\drift}\right) \left(\frac{1-y}{1+y}\right) \nonumber\\
\varrt &=& \left(\frac{\boundarysep}{2\drift^3}\right) \left(\frac{1-2\boundarysep\drift y - y^2}{(1+y)^2}\right) \nonumber
\end{eqnarray}
When $\drift = 0$ (drift rate is zero), the forward equations simplify to $\acc = 0.5$, $\meanrt = \ndt + \boundarysep$, and $\varrt = \boundarysep^2$.

The EZ inverse equations map summary statistics to parameters. Let $y = \log(\acc/(1-\acc))$ be the logit of accuracy. Then,
\begin{eqnarray}
\drift &=& \text{sign}\left(\acc - \frac{1}{2}\right) \sqrt[4]{\frac{y(\acc^2 y - \acc y + \acc - 1/2)}{\varrt}} \nonumber\\
\boundarysep &=& \frac{y}{\drift} \nonumber\\
\ndt &=& \meanrt - \left(\frac{\boundarysep}{2\drift}\right) \left(\frac{1-e^{-\drift\boundarysep}}{1+e^{-\drift\boundarysep}}\right) \nonumber
\end{eqnarray}
{where $\text{sign}(x)$ denotes the sign function, which returns $+1$ if $x \geq 0$ and $-1$ if $x < 0$. In this context, it determines whether drift is positive (when accuracy $\geq 0.5$) or negative (when accuracy $< 0.5$).}

These equations provide point estimates. The bootstrap method extends this to uncertainty quantification.

\subsection{The EZ bootstrap procedure applied to simple diffusion}

The bootstrap exploits the known sampling distributions of the summary statistics. For a sample of size $n$, these distributions are:
\begin{eqnarray*}
n\acc &\sim& \text{Binomial}(n, p)\\
\meanrt &\sim& \mathcal{N}\left(m, \frac{s^2}{n}\right)\\
\varrt &\sim& \text{Gamma}\left(\frac{n-1}{2}, \frac{2s^2}{n-1}\right)
\end{eqnarray*}
where $p$, $m$, and $s^2$ are the true population values of accuracy, mean RT, and RT variance.

\subsection{Distributional assumptions and their justification} The binomial distribution for accuracy is exact. The normal distribution for mean RT is asymptotically valid by the central limit theorem; for finite samples, this approximation is accurate when $n \geq 30$ and the underlying RT distribution has finite variance. The Gamma distribution for variance is exact under the assumption that RTs are normally distributed; while RTs are not exactly normal, this approximation is reasonable for the EZ model when sample sizes are moderate ($n \geq 50$).
Note, however, that we do not make sample size recommendations based on these classical rules of thumb.  The procedure we propose is so fast that Monte Carlo studies are very feasible methods to evaluate the effect of violated assumptions.

\subsection{Independence assumption} We assume approximate independence between the three summary statistics. While these statistics are derived from the same data, they are asymptotically independent under the EZ model assumptions. For finite samples, correlations have negligible impact on coverage properties (see our simulation studies below). 

\subsection{Simulation studies}

We present two standard calibration studies to validate the EZ bootstrap method for the EZ diffusion model.  In both cases, the target of the study was the calibration of the uncertainty quantification in terms of credibility interval coverage -- that is, we sought to confirm that credibility intervals contain the true value at a rate commensurate to their nominal coverage probability (specifically, that 95\% credible intervals contain the true value 95\% of the time).

\subsubsection{Study 1: Single condition}
In Study 1, we evaluated coverage for the regular EZ diffusion model with a single condition. We simulated $n = 100$ trials from true parameters $\boundarysep = 1.0$, $\drift = 0.5$, and $\ndt = 0.2$. For each of 10,000 simulation repetitions, we computed credible intervals at the 95\% confidence level using the EZ bootstrap method (with $1000$ bootstrap samples). 

% \textit{Results.}
We evaluated coverage at the 95\% confidence level across all simulation repetitions. The EZ bootstrap method achieved excellent calibration. Coverage rates were 95.5\%, 94.8\%, and 96.0\% for boundary, drift, and non-decision time, respectively---all within expected sampling variation of the nominal 95\% level. When considering all three parameters simultaneously, the coverage rate was 87.7\%, closely matching the expected value ($0.95^3 \approx 0.857$).

\subsubsection{Study 2: Design matrix extension}
In Study 2, we evaluated coverage for the EZ diffusion model with a design matrix extension. We simulated data from four conditions using a design matrix with three regression weights for each parameter type (boundary, drift, and non-decision time), resulting in 9 beta weight parameters total. The true beta weights were: boundary weights $[1.0, 1.5, 2.0]$, drift weights $[0.4, 0.8, 1.2]$, and non-decision time weights $[0.2, 0.3, 0.4]$. {These parameter values were chosen to represent typical values in the EZ diffusion model \cite{matzke2009psychological}.}

The design matrices for boundary, drift, and non-decision time parameters were:
$$
X_\alpha = \begin{pmatrix}
1 & 0 & 0 \\
0 & 1 & 0 \\
0 & 0 & 1 \\
0 & 0 & 1
\end{pmatrix}\!\!,\,\,\,
X_\delta = \begin{pmatrix}
1 & 0 & 0 \\
0 & 1 & 0 \\
0 & 0 & 1 \\
0 & 1 & 0
\end{pmatrix}\!\!,\,\,\,
X_\tau = \begin{pmatrix}
1 & 0 & 0 \\
0 & 1 & 0 \\
0 & 0 & 1 \\
1 & 0 & 0
\end{pmatrix} 
$$
where each row corresponds to a condition and each column corresponds to a regression weight.
Each condition had $n = 100$ trials. We conducted 10,000 simulation repetitions, computing credible intervals at the 95\% confidence level using EZ bootstrap (with $1000$ bootstrap samples).

The EZ bootstrap method maintained excellent calibration even with the added complexity of the design matrix structure. For boundary weights, coverage rates were 95.7\%, 94.9\%, and 96.6\% across the three weights. For drift weights, coverage rates were 94.6\%, 94.2\%, and 98.0\%. For non-decision time weights, coverage rates were 96.2\%, 94.3\%, and 94.7\%. All coverage rates were within expected sampling variation of the nominal 95\% level.

\subsubsection{Computational performance} 
The EZ bootstrap method is exceptionally computationally efficient. The simulations were run on an off-the-shelf laptop, using a single core of an 11th Gen Intel\textregistered{} Core\texttrademark{} i7-1195G7 processor running at 2.90GHz.  For Study~1, the method required a mean of only 0.16~ms per estimation (SD = 0.02~ms), enabling rapid uncertainty quantification. For Study~2, with the added complexity of the design matrix, the method required a mean of 0.95~ms per estimation (SD = 0.13~ms). A comparable Bayesian analysis reported by \citeauthor{chávez_de_la_peña_vandekerckhove:in_press:hierarchical} (\citeyearNP{chávez_de_la_peña_vandekerckhove:in_press:hierarchical}, Appendix~D) took approximately 3,800~ms. The sub-millisecond computation times of the EZ bootstrap demonstrate that the method makes real-time\footnote{We have a strict definition of ``real-time'' as less than the duration of one display frame in a typical experimental setup -- usually at most 1/120th of a second or 8.33~ms.} uncertainty quantification feasible where traditional methods would be prohibitively slow.


\section{Application to the circular diffusion model}

The circular diffusion model \cite{Smith2016} extends the diffusion model to circular decision spaces, where evidence accumulation occurs in a two-dimensional disk rather than along a one-dimensional line. The EZ version of the circular diffusion model (EZ-CDM; \citeNP{QarehdaghiRad2024}) provides analytical expressions that map four summary statistics---circular mean of choice angles, circular variance of choice angles, mean response time, and response time variance---to four cognitive parameters: drift angle, drift magnitude, boundary radius, and non-decision time. This makes the EZ-CDM framework a synthetic likelihood model: it provides a likelihood function based on summary statistics rather than raw trial data.

\subsection{The EZ circular diffusion model}

The EZ circular diffusion model provides analytical expressions for four summary statistics from a set of $n$ circular decision trials.  Let $\MCA{}$ denote the observed circular mean of choice angles, $\VCA{}$ the observed circular variance of choice angles, $\MRT{}$ the observed mean response time, and $\VRT{}$ the observed response time variance. The model's four parameters are the drift angle $\driftangle$, drift magnitude $\driftmag$, boundary radius $\boundaryradius$, and non-decision time $\ndtcdm$.
This allows us to define forward equations 
$
f_\text{CDM}:
\left(\driftangle, \driftmag, \boundaryradius, \ndtcdm\right)
\mapsto
\left(\MCA{}, \VCA{}, \MRT{}, \VRT{}\right)
$ and inverse equations 
$
i_\text{CDM}:
\left(\MCA{}, \VCA{}, \MRT{}, \VRT{}\right)
\mapsto
\left(\driftangle, \driftmag, \boundaryradius, \ndtcdm\right)
$.

The forward equations $f_\text{CDM}: \left(\driftangle, \driftmag, \boundaryradius, \ndtcdm\right) \mapsto \left(\MCA{}, \VCA{}, \MRT{}, \VRT{}\right)$ map parameters to summary statistics. The choice-angle distribution is von Mises with mean $\mu = \driftangle$ and concentration $\kappa = \boundaryradius\driftmag$.
The circular mean and variance are then:
\begin{eqnarray}
\MCA{} &=& \driftangle \nonumber\\
\VCA{} &=& 1 - R \nonumber
\end{eqnarray}
where $R = {I_1(\kappa)}/{I_0(\kappa)}$ and $I_0$ and $I_1$ are modified Bessel functions of the first kind. The response time moments are:
\begin{eqnarray}
\MRT{} &=& \ndtcdm + \frac{\boundaryradius}{\driftmag} R \nonumber\\
\VRT{} &=& \frac{\boundaryradius^2}{\driftmag^2} R^2 + \frac{2\boundaryradius}{\driftmag^3} R - \boundaryradius^2 \nonumber
\end{eqnarray}

The inverse equations $i_\text{CDM}: \left(\MCA{}, \VCA{}, \MRT{}, \VRT{}\right) \mapsto \left(\driftangle, \driftmag, \boundaryradius, \ndtcdm\right)$ map summary statistics to parameters. The estimation proceeds in four steps:
\begin{eqnarray}
\hat{\driftangle} &=& \MCA{} \nonumber\\
R &=& 1 - \VCA{}, \quad \text{solve } \frac{I_1(\kappa)}{I_0(\kappa)} = R \text{ for } \kappa \nonumber\\
\hat{\driftmag}^2 &=& \frac{-\kappa^2 + \sqrt{\kappa^4 + 4\VRT{}(\kappa^2 R^2 + 2\kappa R)}}{2\VRT{}}, \quad \hat{\boundaryradius} = \frac{\kappa}{\hat{\driftmag}} \nonumber\\
\hat{\ndtcdm} &=& \MRT{} - \frac{\hat{\boundaryradius}}{\hat{\driftmag}}R \nonumber
\end{eqnarray}
where the solution for $\hat{\driftmag}^2$ is obtained by solving the quadratic equation derived from the forward equation for $\VRT{}$.
Solving $R = I_1(k)/I_0(k)$ for $k$ is faciliated by the piecewise estimator of \citeA[see pp.~88-93 for further detail]{Fisher1993}: $$k \approx \begin{cases} 2R + R^{3} + \tfrac{5}{6}R^{5}, & R < 0.53,\\ -0.4 + 1.39R + \dfrac{0.43}{1-R}, & 0.53 \le R < 0.85,\\ \dfrac{1}{R^{3} - 4R^{2} + 3R}, & R \ge 0.85.\end{cases}$$
As an excellent alternative, \citeA{QarehdaghiRad2024} suggest performing a single step of Newton-Raphson optimization. This approach typically provides higher accuracy than the piecewise approximation but requires extra computational effort.

% \textit{Sampling distributions.}
The sampling distributions of the four summary statistics determine the variability and robustness of EZ-CDM parameter estimates. Under the CDM, choice angles follow a von Mises distribution with concentration $\kappa = \boundaryradius\driftmag$, and response times follow the first-passage-time distribution of a two-dimensional Bessel process with drift.

For the circular mean of choice angles (MCA), the angular error $\widehat{\driftangle} - \driftangle$ is approximately normally distributed for large $N$ \cite{Fisher1993}, and so:
\begin{equation}
\widehat{\driftangle} \sim \mathcal{N}\left(\driftangle, \frac{1}{N\kappa R}\right). \nonumber
\end{equation}
The variance decreases rapidly with concentration $\kappa$, making estimation more difficult when drift magnitude or boundary radius are small.

For the circular variance ($\VCA{}$), the mean resultant length $L = \sqrt{\left(\frac{1}{N}\sum_{i=1}^{N} \cos(\theta_i)\right)^2 + \left(\frac{1}{N}\sum_{i=1}^{N} \sin(\theta_i)\right)^2}$ is approximately normal for moderate-to-large $N$, with mean $R$ and variance $\frac{1 - R^2}{N}$ \cite{Fisher1993}. Consequently, $\widehat{\VCA{}} = 1 - R$ is also asymptotically normal with the same variance. The circular variance is only weakly sensitive to outliers in angle (since angles wrap on the circle). However, parameter estimation becomes unstable when $R$ is near zero (because then the data are essentially uniform on the circle and provide limited information about the concentration parameter $\kappa$).

For mean response time ($\MRT{}$), the central limit theorem applies:
\begin{equation}
\widehat{\MRT{}} \sim \mathcal{N}\left(\MRT{}, \frac{\VRT{}}{N}\right) \nonumber
\end{equation}
Mean RT is robust for large $N$ but sensitive to RT contaminants.

For response time variance ($\VRT{}$), the sample variance is asymptotically normal but with slower convergence:
\begin{equation}
\widehat{\VRT{}} \sim \mathcal{N}\left(\VRT{}, \frac{m_4 - \VRT{}^2}{N}\right) \nonumber 
\end{equation}
where $m_4$ is the fourth central moment of the RT distribution. This statistic is the most contamination-sensitive in EZ-CDM, as a small number of extreme RTs can cause large upward shifts (see \citeNP{chávez_de_la_peña_etal:preprint:hierarchical}, and \citeNP{QarehdaghiRad2024}, for approaches using robust summary statistics).

The EZ bootstrap procedure can be applied to EZ-CDM by resampling these statistics from their asymptotic distributions. The angle-based statistics ($\MCA{}$ and $\VCA{}$) have well-behaved sampling distributions, while the RT-based statistics ($\MRT{}$ and $\VRT{}$) introduce most of the sampling variability in the inversion system, especially when solving for drift magnitude and boundary radius.

\subsection{Simulation studies}

We again present two calibration studies to validate the EZ bootstrap method for the EZ circular diffusion model.  The target of the study was again the calibration of the uncertainty quantification in terms of credibility interval coverage (95\% credibility).

\subsubsection{Study 1: Single condition}
In Study 1, we evaluated coverage for the regular EZ circular diffusion model with a single condition. We simulated $n = 100$ trials from true parameters $\driftangle = 0.0$, $\driftmag = 1.0$, $\boundaryradius = 1.0$, and $\ndtcdm = 0.2$. For each of 10,000 simulation repetitions, we computed credible intervals at the 95\% confidence level using the EZ bootstrap method (with $1000$ bootstrap samples). 

% \textit{Results.}
We evaluated coverage at the 95\% confidence level across 10,000 simulation repetitions. The EZ bootstrap method achieved excellent calibration. Coverage rates were 93.9\%, 94.3\%, 95.2\%, and 94.8\% for drift angle, drift magnitude, boundary radius, and non-decision time, respectively---all within expected sampling variation of the nominal 95\% level. When considering all four parameters simultaneously, the coverage rate was 82.4\%, closely matching the expected value ($0.95^4 \approx 0.815$).

\subsubsection{Study 2: Design matrix extension}
In Study 2, we evaluated coverage for the EZ circular diffusion model with a design matrix extension. We simulated data from four conditions using a design matrix with three regression weights for each parameter type (drift angle, drift magnitude, boundary radius, and non-decision time), resulting in 12 beta weight parameters total. The true beta weights were: drift angle weights $[0.0, 0.5, 1.0]$, drift magnitude weights $[0.5, 0.7, 0.9]$, boundary radius weights $[0.8, 1.0, 1.2]$, and non-decision time weights $[0.15, 0.2, 0.25]$ \cite{villarreal_etal:2024:graphical}.

The design matrices for drift angle, drift magnitude, boundary radius, and non-decision time parameters were:
$$
X_{\driftangle} = \begin{pmatrix}
1 & 0 & 0 \\
0 & 1 & 0 \\
0 & 0 & 1 \\
0 & 1 & 0
\end{pmatrix}\!\!,\,\,\,
X_{\driftmag} = \begin{pmatrix}
1 & 0 & 0 \\
0 & 1 & 0 \\
0 & 0 & 1 \\
0 & 1 & 0
\end{pmatrix}\!\!,$$
$$X_{\boundaryradius} = \begin{pmatrix}
1 & 0 & 0 \\
0 & 1 & 0 \\
0 & 0 & 1 \\
1 & 0 & 0
\end{pmatrix}\!\!,\,\,\,
X_{\ndtcdm} = \begin{pmatrix}
1 & 0 & 0 \\
0 & 1 & 0 \\
0 & 0 & 1 \\
0 & 0 & 1
\end{pmatrix} 
$$
where each row corresponds to a condition and each column corresponds to a regression weight.
Each condition had $n = 100$ trials. We conducted 10,000 simulation repetitions, computing credible intervals at the 95\% confidence level using EZ bootstrap (with $1000$ bootstrap samples).

The method maintained excellent calibration: For drift angle weights, coverage rates were 89.5\%, 94.5\%, and 94.2\% across the three weights. For drift magnitude weights, coverage rates were 94.5\%, 94.4\%, and 89.3\%. For boundary radius weights, coverage rates were 95.8\%, 94.8\%, and 93.0\%. For non-decision time weights, coverage rates were 95.8\%, 94.6\%, and 95.7\%. All coverage rates were within expected sampling variation of the nominal 95\% level.

\subsubsection{Computational performance}
For Study~1, the method required a mean of only 0.33~ms per estimation (SD = 0.03~ms), enabling rapid uncertainty quantification. For Study~2, with the added complexity of the design matrix, the method required a mean of 1.26~ms per estimation (SD = 0.19~ms). 

\section{Other models}

The EZ bootstrap method generalizes to any model in the EZ class -- that is, any model with analytical forward and inverse equations smoothly mapping parameters to summary statistics. Here we briefly describe one additional example: the one-high-threshold source-monitoring {multinomial processing tree (MPT)} model.

\subsection{One-high-threshold source-monitoring model}

The one-high-threshold source-monitoring MPT model \cite{BatchelderRiefer1990,BatchelderRiefer1999} is a multinomial processing tree model for source memory that allows participants to identify items as coming from Source~A, Source~B, or as new. The model has four parameters: detection probability $\Detect$, source discrimination probability $\Discrim$, criterion for new items $\Criterion$, and guessing probability $\Guess$.

The forward equations map parameters to observed response proportions. For old items from Source~A, the forward equations are:
\begin{eqnarray}
\AAstat &=& \Detect\Discrim + (1-\Detect)\Guess \nonumber\\
\BAstat &=& \Detect(1-\Discrim) + (1-\Detect)(1-\Guess) \nonumber
\end{eqnarray}
where $\AAstat = P(\text{``A''} \mid \text{old from A})$ and $\BAstat = P(\text{``B''} \mid \text{old from A})$. Analogous expressions hold for old items from Source~B:
\begin{eqnarray}
\ABstat &=& \Detect(1-\Discrim) + (1-\Detect)\Guess \nonumber\\
\BBstat &=& \Detect\Discrim + (1-\Detect)(1-\Guess) \nonumber
\end{eqnarray}
where $\ABstat = P(\text{``A''} \mid \text{old from B})$ and $\BBstat = P(\text{``B''} \mid \text{old from B})$. For new items, the model specifies:
\begin{eqnarray}
\NNstat &=& \Criterion \nonumber\\
\ANstat &=& (1-\Criterion)\Guess \nonumber\\
\BNstat &=& (1-\Criterion)(1-\Guess) \nonumber
\end{eqnarray}
where $\NNstat = P(\text{``new''} \mid \text{new})$, $\ANstat = P(\text{``A''} \mid \text{new})$, and $\BNstat = P(\text{``B''} \mid \text{new})$.
The inverse equations provide closed-form parameter estimates:
\begin{eqnarray}
\Criterion &=& \NNstat \nonumber\\
\Guess &=& \frac{\ANstat}{1-\NNstat} \nonumber\\
\Detect &=& \frac{\AAstat + \ABstat - 2\Guess}{1-2\Guess} \nonumber\\
\Discrim &=& \frac{\AAstat - (1-\Detect)\Guess}{\Detect} \nonumber
\end{eqnarray}
These linear relationships allow closed-form inversion, illustrating a simple but nontrivial MPT for which both forward and inverse transformations are available in closed form.

The sampling distributions for MPT models are multinomial: the observed counts in each response category follow a multinomial distribution with probabilities given by the forward equations. For old items from Source~A, the counts $(n_{\AAstat}, n_{\BAstat})$ follow:
\begin{equation}
(n_{\AAstat}, n_{\BAstat}) \sim \text{Multinomial}(n_A, (\AAstat, \BAstat)) \nonumber
\end{equation}
where $n_A$ is the number of old items from Source~A and $\AAstat + \BAstat = 1$. Similarly, for old items from Source~B:
\begin{equation}
(n_{\ABstat}, n_{\BBstat}) \sim \text{Multinomial}(n_B, (\ABstat, \BBstat)) \nonumber
\end{equation}
where $n_B$ is the number of old items from Source~B and $\ABstat + \BBstat = 1$. For new items:
\begin{equation}
(n_{\NNstat}, n_{\ANstat}, n_{\BNstat}) \sim \text{Multinomial}(n_N, (\NNstat, \ANstat, \BNstat)) \nonumber
\end{equation}
where $n_N$ is the number of new-item trials and $\NNstat + \ANstat + \BNstat = 1$. The EZ bootstrap procedure can be applied by resampling from these multinomial distributions and transforming through the inverse equations to obtain bootstrap parameter estimates.

We provide this solvable MPT example only as an illustration. The class of MPT models with tractable analytical inverses is rather narrow and does not reflect the complexity of most MPT applications, where estimation usually requires numerical optimization or Bayesian inference \cite{matzke_etal:2018:nonstandard}.

\section{Discussion}

The EZ bootstrap method provides fast uncertainty quantification. It maintains proper calibration while achieving dramatic speed improvements over previous methods such as the Bayesian implementation by \citeA{chávez_de_la_peña_vandekerckhove:in_press:hierarchical} -- the method achieves coverage rates that closely match nominal levels for both simple single-condition models and more complex design matrix extensions.

The method's success relies on the synthetic likelihood structure of EZ models. By operating directly on summary statistics with known sampling distributions, we avoid the computational cost of raw data resampling while preserving statistical validity. The transformation-of-variables approach generalizes to any model in the EZ class -- that is, any model with invertible mappings from parameters to summary statistics. The calibration results illustrate that our method maintains proper statistical properties while delivering computational efficiency.


\subsection{Limitations}

{The bootstrap method has a few limitations worth pointing out. First, the method is limited to the class of EZ models and their design matrix extensions. While identifiable models will typically \textit{have} both the forward and inverse equations, the equations are often not known or are mathematically inconvenient. Second, the method requires that summary statistics have known or approximately known sampling distributions. Approximations such as the normal distribution for mean RT and the Gamma for variance are asymptotically valid but may be less accurate for very small samples ($n < 30$) or when RT distributions are highly skewed, but good approximations of sampling distributions are known for very many summary statistics. Third, we assume approximate independence between summary statistics. While this holds asymptotically and is approximately valid for sufficiently large samples, correlations may be non-negligible for very small samples or models beyond our consideration.}


\subsection{Future directions}

Several extensions are possible. The method could be extended to yet more models with known mappings from parameters to summary statistics. Hierarchical extensions could enable multi-level modeling with summary statistics. Integration with experimental platforms could enable real-time semi-Bayesian design optimization and computerized adaptive tests.

Perhaps most promisingly, implicit mappings from parameters to summary statistics exist even when explicit equations are unknown. Techniques from statistical learning, such as normalizing flow neural networks, could be used to approximate these mappings for use in the synthetic bootstrap method, extending the approach to a broader class of models.


\section{Computational reproducibility}

All code associated with this project is available via GitHub. The simulation studies reported in this paper can be reproduced using the C implementation available in the GitHub repository \texttt{joachimvandekerckhove/ez-bootstrap}. The implementation requires the GNU Scientific Library (GSL) for random number generation and statistical distributions.

To reproduce the simulation studies, first
(1)~clone the repository, then
(2)~install dependencies if needed (e.g., install \texttt{libgsl-dev} to obtain GSL development libraries on Debian/Ubuntu systems), and 
(3)~run the simulation script
\texttt{./run\_all.sh}.

The script executes both simulation studies for the EZ-DDM and EZ-CDM with the following editable parameters: 10,000 simulation repetitions, sample size $n = 100$ trials per condition, and 1,000 bootstrap samples per repetition. The simulation outputs include coverage rates for each parameter (or beta weight) and computational performance metrics.

{A Python implementation of the EZ-DDM is also available as part of the larger EZAS package (\url{https://github.com/joachimvandekerckhove/ezas}), which uses the same method but is not as fast as the C implementation.}


\ifdefined\apaformat

\newcommand{\dec}[2]{\noindent\textbf{#1} --- #2}

\section{Declarations}

\dec{Funding}{This work was supported by the National Research Council.}
   
\dec{Competing interests}{The authors report no competing interest.}

\dec{Ethics approval}{N/A}

\dec{Consent to participate}{N/A}

\dec{Consent for publication}{N/A}

\dec{Availability of data and materials}{N/A}

\dec{Code availability}{See Section ``Computational reproducibility.''}

\fi


\bibliography{includes/references,includes/cidlab}\null

\subsection{Disclaimer}
The views expressed are those of the authors and do not necessarily reflect the official policy or position of the Department of the Air Force, the Department of Defense, or the U.S. Government. 


% \appendix
% \ifdefined\apaformat
% \section{Reproducing the simulation studies}
% \else
% \section{Appendix: Reproducing the simulation studies}
% \fi

\end{document}
"
% To compile in original format: pdflatex main.tex

% \def\apaformat{1}

\ifdefined\apaformat
  % APA-style preamble for draft format
% This file is included by main.tex when compiling in APA format

\documentclass[man,12pt]{apa6}

\usepackage{
  amsmath,amssymb,amsfonts,
  csquotes,
  epsfig,epstopdf,epigraph,
  fancyhdr,
  graphicx,
  hyperref,
  mathtools,
  setspace,
  subfig,
  tikz,
  url,
  apacite}

% Set up bibliography style for APA format
\bibliographystyle{apacite}

% Redefine cidlab-specific commands to work with apa6
\newcommand{\leadauthor}[1]{}
\newcommand{\significancestatement}[1]{}
\newcommand{\authorcontributions}[1]{}
\newcommand{\authordeclaration}[1]{}
\newcommand{\equalauthors}[1]{}
\newcommand{\correspondingauthor}[1]{}
\newcommand{\dates}[1]{}
\newcommand{\verticaladjustment}[1]{}
\newcommand{\firststyle}{fancy}
\newcommand{\abscontentformatted}{}
\newcommand{\abscontent}{}
\newcommand{\nocaption}{}

% Redefine section commands for APA format
\let\origsection\section
\let\origsubsection\subsection
\renewcommand{\section}[1]{\origsection{#1}}
\renewcommand{\subsection}[1]{\origsubsection{#1}}

% Redefine citation commands for APA format
% \citeNP is already defined by apacite

% Define introduction command for APA format
\newcommand{\introduction}[1]{\section{Introduction}\label{sec:introduction}}


% Include shared metadata
% Shared metadata for both document formats
% This file is included by both draft.tex and pretty.tex

% Title, authors and affiliations - format-specific
\ifdefined\apaformat
  \title{Fast Uncertainty Quantification in EZ Cognitive Models}
  % APA-specific formatting
  \shorttitle{EZ Cognitive Models}

  % APA format uses combined author/affiliation
  \author{Joachim Vandekerckhove$^{a,\dagger}$ and Elizabeth L.\ Fox$^{b}$\vspace{3ex}}
  \affiliation{%
  $^a$ University of California, Irvine\\
  $^b$ Air Force Research Laboratory\\\vspace{3ex}
  $^\dagger$ To whom correspondence should be addressed. E-mail: joachim@uci.edu.}
\else
  % cidlab format uses individual author/affiliation with symbols
  \title{Fast uncertainty quantification in EZ cognitive models}

  \author[a,*]{Joachim Vandekerckhove}
  \author[b]{Elizabeth L.\ Fox}

  \affil[a]{University of California, Irvine}
  \affil[b]{Air Force Research Laboratory}

  \leadauthor{Vandekerckhove} 
    
   \authorcontributions{All authors contributed to the final draft.}
   \authordeclaration{This work was supported by a National Research Council Fellowship to JV.}
   \correspondingauthor{\textsuperscript{*}Correspondence concerning this article should be addressed to Joachim Vandekerckhove (joachim@uci.edu).}
   \equalauthors{}
    
   \dates{Draft of \today{}.}%  Planned outlet: \textit{Advances in Methods and Practices in Psychological Science}.}
    
\fi

% Abstract - format-specific
\ifdefined\apaformat
  % APA format uses \abstract command
  \abstract{%
    \input{includes/abstract}
  }
\fi

% Keywords
\keywords{cognitive modeling, diffusion model, bootstrap, uncertainty quantification, EZ diffusion}

\graphicspath{{./figures/}}

\let\citep\cite
\let\citet\citeA


\else
  % Original cidlab-style preamble for pretty format
% This file is included by main.tex when compiling in original format

\documentclass[9pt,twocolumn,twoside]{cidlab/cidlab}

\usepackage{
  amsmath,amssymb,amsfonts,
  csquotes,
  epsfig,epstopdf,epigraph,
  graphicx,
  mathtools,
  subfig,
  tikz,
  url}
\usepackage[super]{nth}
\usepackage[english]{babel}

% Define introduction command that does nothing in cidlab format
\newcommand{\introduction}[1]{}


% Abstract for cidlab format
\begin{abstract} 


Classical approaches to uncertainty quantification in cognitive modeling rely on computationally expensive Monte Carlo methods or resampling of raw trial data, which creates a barrier for real-time analysis and large-scale studies. We present a computationally efficient bootstrap method that operates directly on summary statistics, exploiting the synthetic likelihood structure of a small class of cognitive models that includes the simple diffusion model and the circular diffusion model in addition to signal detection and multinomial processing trees. The method relies on a numerical transformation-of-variables technique in which known sampling distributions of summary statistics of behavior are parametrically bootstrapped, after which the resampled statistics are transformed to parameter estimates with a known analytical system. This approach does not require additional assumptions beyond those already made by the models themselves, but achieves over 1000-fold speed improvements over already efficient fully Bayesian methods. The proposed method makes real-time uncertainty quantification accessible and enables new applications in adaptive testing, meta-analyses, and exploratory data analysis.
\end{abstract}

\significancestatement{}
\authorcontributions{}
\authordeclaration{}
\equalauthors{}
\correspondingauthor{\textsuperscript{$\dagger$}To whom correspondence should be addressed. E-mail: joachim@uci.edu.}

\dates{Draft of \today.}
% \dates{Planned outlet: \textit{Nature Human Behaviour}.}

% Include shared metadata
% Shared metadata for both document formats
% This file is included by both draft.tex and pretty.tex

% Title, authors and affiliations - format-specific
\ifdefined\apaformat
  \title{Fast Uncertainty Quantification in EZ Cognitive Models}
  % APA-specific formatting
  \shorttitle{EZ Cognitive Models}

  % APA format uses combined author/affiliation
  \author{Joachim Vandekerckhove$^{a,\dagger}$ and Elizabeth L.\ Fox$^{b}$\vspace{3ex}}
  \affiliation{%
  $^a$ University of California, Irvine\\
  $^b$ Air Force Research Laboratory\\\vspace{3ex}
  $^\dagger$ To whom correspondence should be addressed. E-mail: joachim@uci.edu.}
\else
  % cidlab format uses individual author/affiliation with symbols
  \title{Fast uncertainty quantification in EZ cognitive models}

  \author[a,*]{Joachim Vandekerckhove}
  \author[b]{Elizabeth L.\ Fox}

  \affil[a]{University of California, Irvine}
  \affil[b]{Air Force Research Laboratory}

  \leadauthor{Vandekerckhove} 
    
   \authorcontributions{All authors contributed to the final draft.}
   \authordeclaration{This work was supported by a National Research Council Fellowship to JV.}
   \correspondingauthor{\textsuperscript{*}Correspondence concerning this article should be addressed to Joachim Vandekerckhove (joachim@uci.edu).}
   \equalauthors{}
    
   \dates{Draft of \today{}.}%  Planned outlet: \textit{Advances in Methods and Practices in Psychological Science}.}
    
\fi

% Abstract - format-specific
\ifdefined\apaformat
  % APA format uses \abstract command
  \abstract{%
    \input{includes/abstract}
  }
\fi

% Keywords
\keywords{cognitive modeling, diffusion model, bootstrap, uncertainty quantification, EZ diffusion}

\graphicspath{{./figures/}}

\let\citep\cite
\let\citet\citeA


\AtBeginDocument{
\verticaladjustment{-2pt}
\thispagestyle{firststyle}
\ifthenelse{\boolean{shortarticle}}{\ifthenelse{\boolean{singlecolumn}}{\abscontentformatted}{\abscontent}}{}
}

\fi
% Main document content starts here


\setlength {\marginparwidth }{2cm}


\usepackage{todonotes}

\newif\ifshownotes

\newcommand{\definenote}[2]{%
  \expandafter\newcommand\csname #1\endcsname[1]{%
    \ifshownotes
      \todo[inline,color=#2]{##1 -\textit{#1}}%
    \fi
  }%
}

\shownotesfalse

% https://en.wikipedia.org/wiki/X11_color_names
\definenote{jv}{Lime}
\definenote{elf}{Gold}


% Comment this out to remove notes:
\shownotestrue

% Macros for EZ model summary statistics and parameters
% This file contains all custom macros used throughout the document

% Signal Detection Theory (SDT) parameters
\newcommand{\dprime}{d'}
\newcommand{\criterion}{c}

% Signal Detection Theory (SDT) summary statistics
\newcommand{\HitRate}{H}
\newcommand{\FalseAlarm}{F}

% EZ diffusion model parameters
\newcommand{\drift}{\delta}
\newcommand{\boundarysep}{\alpha}
\newcommand{\ndt}{\tau}

% EZ diffusion model summary statistics
\newcommand{\acc}{\hat{P}}
\newcommand{\meanrt}{\hat{M}}
\newcommand{\varrt}{\hat{V}}

% EZ circular diffusion model (EZ-CDM) parameters
\newcommand{\driftangle}{\theta_v}
\newcommand{\driftmag}{v}
\newcommand{\boundaryradius}{a}
\newcommand{\ndtcdm}{t_0}

% EZ circular diffusion model (EZ-CDM) summary statistics
\newcommand{\MCA}{M_a}
\newcommand{\VCA}{V_a}
\newcommand{\MRT}{M_t}
\newcommand{\VRT}{V_t}

% One-high-threshold source-monitoring MPT model parameters
\newcommand{\Detect}{D}
\newcommand{\Discrim}{d}
\newcommand{\Criterion}{c}
\newcommand{\Guess}{g}

% One-high-threshold source-monitoring MPT model summary statistics
\newcommand{\AAstat}{A_A}
\newcommand{\BAstat}{B_A}
\newcommand{\ABstat}{A_B}
\newcommand{\BBstat}{B_B}
\newcommand{\NNstat}{N_N}
\newcommand{\ANstat}{A_N}
\newcommand{\BNstat}{B_N}


\usepackage{algorithm}
\usepackage{algpseudocode}
\usepackage{listings}
\lstset{extendedchars=true,upquote=true}
% Handle UTF-8 characters in listings using literate option
\lstdefinestyle{utf8style}{
  extendedchars=true,
  upquote=true,
  literate={≈}{{$\approx$}}1
}
\lstset{style=utf8style}

\renewcommand{\subsubsection}[1]{\textit{#1}.}

\begin{document}

\maketitle

\introduction{}

\noindent{}%
Cognitive models serve dual purposes in psychological research. On one hand, they can be used to understand the cognitive infrastructure---the underlying mechanisms and processes that give rise to behavior. On the other hand, cognitive models can function as measurement tools, translating behavioral observations into interpretable psychological parameters. This paper focuses on the latter: we develop computationally efficient methods for uncertainty quantification in cognitive models used as measurement instruments. The need for computational efficiency becomes critical in large-scale applications such as meta-analyses across hundreds of studies, real-time experimental analysis during data collection, and individual differences research requiring parameter estimation for thousands of participants. Traditional approaches to uncertainty quantification, which rely on computationally expensive resampling or Bayesian inference, create barriers in these contexts. We propose a bootstrap method that exploits the synthetic likelihood structure of a class of cognitive models we call `EZ' models, achieving dramatic speed improvements while maintaining statistical validity.

\section{Synthetic likelihood methods}

Synthetic likelihood methods provide a framework for statistical inference when the likelihood function is intractable but summary statistics with known or estimable distributions are available \cite{wood_statistical_2010,Price2017}. The core idea is to replace the full likelihood with a likelihood based on summary statistics, which can be computed efficiently even when the full data likelihood cannot.

The synthetic likelihood approach was introduced by \citeauthor{wood_statistical_2010} for inference in complex ecological systems with chaotic dynamics. When direct likelihood evaluation is impossible, one can extract phase-insensitive summary statistics from the data and compare them to summary statistics simulated from the model. This approach has been extended to Bayesian settings \cite{Price2017}, where the synthetic likelihood is used within a Bayesian framework to enable posterior inference.

Certain cognitive models, which we will refer to here as the `EZ' models (after the EZ diffusion model; \citeNP{WagenmakersEtAl2007}), similarly fit the description of synthetic likelihood models, and provide a natural application of methods specific to that class. For example, in the EZ diffusion model, rather than working with the full likelihood over all trial-level data, we work with three summary statistics: accuracy rate, mean response time, and response time variance. These statistics have known sampling distributions, allowing us to construct a likelihood function directly from the summaries \cite{chávez_de_la_peña_vandekerckhove:in_press:hierarchical}. This synthetic likelihood approach is particularly powerful because the EZ model provides analytical expressions that map parameters to these summary statistics, eliminating the need for simulation, and paving the way for uncertainty quantification using a variation on a classic transformation-of-variables technique, which we will describe after introducing the model class.

\section{The class of EZ models}

We will rely on a class of cognitive models we call `EZ'.  EZ models are defined by the existence of an invertible system of equations that maps a model parameter vector $\theta$ to a set of summary statistics $S$.  The vector $\theta$ need not contain \textit{all} of the model's parameters, but may be limited to a subset of interest (as in the application by \citeNP{wood_statistical_2010}).  The system consists of a set of smooth `forward' equations $f$ and `inverse' equations $i$:
\begin{eqnarray}
    f& :&  \theta \mapsto S \nonumber\\
    i& :&  S \mapsto \theta\label{eq:generic:inv}
\end{eqnarray}

The EZ class contains some well-known cognitive models.  We take signal detection theory (SDT; \citeNP{swets_green:1966:signal}) as an example:
\begin{eqnarray}
    (\HitRate,\FalseAlarm) &=& f_\text{SDT}(\dprime,\criterion) \nonumber \\
    &\Leftrightarrow& \left\{
    \begin{array}{rcl}
         \HitRate&=& \Phi(\dprime - \criterion) \\
         \FalseAlarm&=& \Phi(-\criterion) \\
    \end{array}\right.
\end{eqnarray}
where $\Phi$ denotes the standard normal cumulative distribution function. The inverse equations map summary statistics back to parameters:
\begin{eqnarray}
    (\dprime,\criterion) &=& i_\text{SDT}(\HitRate,\FalseAlarm) \nonumber \\
    &\Leftrightarrow& \left\{
    \begin{array}{rcl}
         \dprime&=& \Phi^{-1}(\HitRate) - \Phi^{-1}(\FalseAlarm) \\
         \criterion&=& -\Phi^{-1}(\FalseAlarm) \\
    \end{array}\right.
\end{eqnarray}
where $\Phi^{-1}$ denotes the inverse standard normal cumulative distribution function. Here, $\HitRate$ denotes the hit rate (probability of correctly identifying a signal), $\FalseAlarm$ denotes the false alarm rate (probability of incorrectly identifying a signal when none is present), $\dprime$ denotes sensitivity (the ability to discriminate signal from noise), and $\criterion$ denotes the criterion (the decision threshold).

Other members of the EZ class include the EZ diffusion model (\citeNP{WagenmakersEtAl2007}), the EZ circular diffusion model \cite{QarehdaghiRad2024}, and some simple multinomial processing tree models with analytical solutions. {These models are discussed in detail below.}


\section{The EZ bootstrap procedure}\label{sec:ez_bootstrap}

We propose a bootstrap method we call `EZ bootstrap' that exploits the synthetic likelihood structure of EZ models. The method operates directly on summary statistics, using their known sampling distributions to perform a parametric bootstrap. 

In the abstract, EZ bootstrap proceeds along the following steps. First, (1) compute summary statistics $S$ from the {observed} data. Then, (2) for each bootstrap repetition $b = 1, \ldots, B$: (2a) sample summary statistics $S^*_b$ from their known sampling distributions $\mathcal{D}_S$ (the requirement that these distributions be known or estimable is fundamental to the method), and (2b) transform $S^*_b$ through the inverse equations $i$ to obtain bootstrap parameter estimates $\theta^*_b = i(S^*_b)$. Finally, (3) aggregate the bootstrap parameter estimates $\{\theta^*_1, \ldots, \theta^*_B\}$ to compute means, standard deviations, and credible intervals.

This procedure is formalized in Algorithm~\ref{alg:ez_bootstrap}. EZ bootstrap is fundamentally a transformation-of-variables approach to uncertainty estimation. Rather than resampling raw data and re-estimating parameters, we resample the summary statistics from their known distributions, then transform these resampled statistics through the inverse equations to obtain parameter estimates. {Specifically, in Algorithm~\ref{alg:ez_bootstrap} we sample $S^*_b$ from $\mathcal{D}_S$ for each bootstrap repetition $b = 1, \ldots, B$, then compute $\theta^*_b = i(S^*_b)$ to obtain the bootstrap parameter estimates $\{\theta^*_1, \ldots, \theta^*_B\}$.} This transformation preserves the statistical properties of the bootstrap while eliminating the computational cost of data resampling. 

\subsection{Transformation of variables for uncertainty quantification}
Transformation of variables is a fundamental method in statistics for propagating uncertainty through deterministic transformations (\citeNP{rice2006mathematical}, see also \citeNP{hayes2013relative}). When a random variable $X$ with known distribution is transformed through a function $Y = g(X)$, the distribution of $Y$ can be derived using the change-of-variables formula. This technique is central to the EZ bootstrap method.

In the context of uncertainty quantification, transformation of variables allows us to derive the distribution of parameter estimates from the distribution of summary statistics in EZ models. The inverse equations (Eq.~\ref{eq:generic:inv}) provide a deterministic transformation from summary statistics to parameters. By applying this transformation to bootstrap samples of the summary statistics, we obtain bootstrap samples of the parameters, which can then be used to quantify uncertainty.

The key advantage of this approach is computational efficiency. Rather than resampling raw data and re-estimating parameters for each bootstrap sample---which would require expensive optimization or simulation \cite{efron1994introduction}---we resample the summary statistics from their known distributions and apply the analytical transformation, {as formalized in Algorithm~\ref{alg:ez_bootstrap}}. This eliminates the computational bottleneck while preserving the statistical properties of the bootstrap.

The transformation-of-variables approach is made possible in EZ models particularly because the inverse transformation is analytical, smooth, and one-to-one (within the parameter space, or a relevant subspace thereof). 

\begin{algorithm}[bt]
\caption{EZ Bootstrap Procedure}
\label{alg:ez_bootstrap}
\begin{algorithmic}[1]
\Require Observed summary statistics $S$, number of bootstrap samples $B$, inverse transformation $i: S \mapsto \theta$, sampling distributions $\mathcal{D}_S$ for each component of $S$
\Ensure Bootstrap parameter estimates $\{\theta^*_1, \ldots, \theta^*_B\}$
\State $\Theta^* \leftarrow \emptyset$
\For{$b = 1$ to $B$}
    \State Sample $S^*_b \sim \mathcal{D}_S$ from known sampling distributions of $S$
    \State $\theta^*_b \leftarrow i(S^*_b)$
    \State $\Theta^* \leftarrow \Theta^* \cup \{\theta^*_b\}$
\EndFor
\Return $\Theta^*$
\end{algorithmic}
\end{algorithm}

\subsection{Theoretical justification for coverage preservation} The bootstrap procedure maintains proper coverage properties when the inverse transformation $i: S \mapsto \theta$
is a continuous, differentiable function. Under the bootstrap sampling scheme, we generate $S^*$ from its known distribution, then compute $\theta^* = i(S^*)$. Since $i$ is deterministic and the bootstrap samples $S^*$ have the correct sampling distributions, the transformed samples $\theta^*$ correctly represent the sampling distribution of the parameter estimates. This follows from the invariance property of bootstrap methods under smooth transformations \cite{rice2006mathematical}. The coverage properties are preserved because the bootstrap distribution of the transformed parameters matches the true sampling distribution of the parameter estimates -- up to the accuracy of the assumed distributions for the summary statistics.

\section{Design matrix extension}

In applied contexts, we often constrain parameters across conditions. These constraints can be expressed using design matrices. For example, {for diffusion models,} if drift rates, {$\drift$,} vary linearly with a stimulus property $x_i$, we can write:
$$\drift_i = \beta_0 + \beta_1 x_i.$$
If they vary quarilinearly, then:
$$\drift_i = \beta_0 + \beta_1 x_i + \beta_2 x_i^2.$$

This generalizes, so that parameter constraints can be expressed as:
$$\Delta = X \times \boldsymbol{\beta}$$
where $\Delta$ is a vector of parameters, $X$ is a design matrix, and $\boldsymbol{\beta}$ is a vector of regression weights.

If the model parameters $\Delta$ are known and the model is identified, the weights $\boldsymbol{\beta}$ can be derived via matrix inversion:
\begin{equation}
    \hat{\boldsymbol{\beta}} = X^{-1} \times \Delta \label{eq:design:solution}
\end{equation}

Importantly, adding the design matrix extension \textit{keeps the model inside the EZ class} and does not change the bootstrap logic.  This is easy to prove: Let $i: S \mapsto \theta$ denote the EZ inverse transformation mapping summary statistics $S$ to parameters $\theta$ (Eq.~\ref{eq:generic:inv}), and let $h: \theta \mapsto \boldsymbol{\beta}$ denote the design matrix transformation mapping parameters $\theta$ to beta weights $\boldsymbol{\beta}$ via $h(\theta) = X^{-1}\theta$ (where $X$ is the design matrix). The basic EZ bootstrap generates bootstrap samples $\theta^* = i(S^*)$ where $S^*$ is resampled from the known distribution of summary statistics. With the design matrix extension, we generate bootstrap samples $\boldsymbol{\beta}^* = h(i(S^*)) = X^{-1}i(S^*)$. 
Since $i$ is a deterministic, smooth transformation (by the definition of EZ models) and $h$ is a deterministic linear transformation (matrix multiplication), their composition $\left(h \circ i\right): S \mapsto \boldsymbol{\beta}$ is also deterministic and smooth. By the same invariance property of bootstrap methods under smooth transformations \cite{rice2006mathematical}, if the bootstrap distribution of $S^*$ correctly represents the sampling distribution of $S$, then the bootstrap distribution of $\boldsymbol{\beta}^* = (h \circ i)(S^*)$ correctly represents the sampling distribution of $\boldsymbol{\beta} = (h \circ i)(S)$. Therefore, the design matrix extension preserves the bootstrap's coverage properties: the model remains in the EZ class because we still use the EZ inverse transformation, and the bootstrap logic is unchanged because we are still propagating uncertainty through a deterministic transformation.

\section{Application to the simple diffusion model}

The drift diffusion model (DDM) provides a unified account of decision-making processes, modeling both response accuracy and response time as the result of an evidence accumulation process \cite{Ratcliff1978,Wagenmakers2009Bertelson}. This framework has proven valuable in cognitive psychometrics, enabling the translation of behavioral patterns into interpretable cognitive parameters \cite<e.g.,>{pe_etal:2013:relationship,schubert_etal:2019:neuroscience,vandekerckhove:2014:simultaneous}.

The EZ diffusion model \cite{WagenmakersEtAl2007} was originally proposed as a method to simplify use of the DDM by providing analytical expressions that map three summary statistics---accuracy rate, mean response time, and response time variance---to three cognitive parameters: drift rate, boundary separation, and non-decision time. This makes the EZ framework a synthetic likelihood model: it provides a likelihood function based on summary statistics rather than raw trial data.

\subsection{The EZ diffusion model}

The EZ diffusion model provides analytical expressions for three summary statistics from a set of $n$ two-choice decision trials.  Let $\acc$ denote the observed accuracy rate, $\meanrt$ the observed mean response time, and $\varrt$ the observed response time variance. The model's three parameters are the drift rate $\drift$, boundary separation $\boundarysep$, and non-decision time $\ndt$.
This allows us to define forward equations 
$
f_\text{DIF}:
\left(\drift,\boundarysep,\ndt\right)
\mapsto
\left(\acc, \meanrt, \varrt\right)
$ and inverse equations 
$
i_\text{DIF}:
\left(\acc, \meanrt, \varrt\right)
\mapsto
\left(\drift,\boundarysep,\ndt\right)
$.

The EZ forward equations map parameters to summary statistics. Let $y = \exp(-\boundarysep \drift)$ where $\boundarysep$ is boundary separation and $\drift$ is drift rate. Then,
\begin{eqnarray}
\acc &=& \frac{1}{1+y} \nonumber\\
\meanrt &=& \ndt + \left(\frac{\boundarysep}{2\drift}\right) \left(\frac{1-y}{1+y}\right) \nonumber\\
\varrt &=& \left(\frac{\boundarysep}{2\drift^3}\right) \left(\frac{1-2\boundarysep\drift y - y^2}{(1+y)^2}\right) \nonumber
\end{eqnarray}
When $\drift = 0$ (drift rate is zero), the forward equations simplify to $\acc = 0.5$, $\meanrt = \ndt + \boundarysep$, and $\varrt = \boundarysep^2$.

The EZ inverse equations map summary statistics to parameters. Let $y = \log(\acc/(1-\acc))$ be the logit of accuracy. Then,
\begin{eqnarray}
\drift &=& \text{sign}\left(\acc - \frac{1}{2}\right) \sqrt[4]{\frac{y(\acc^2 y - \acc y + \acc - 1/2)}{\varrt}} \nonumber\\
\boundarysep &=& \frac{y}{\drift} \nonumber\\
\ndt &=& \meanrt - \left(\frac{\boundarysep}{2\drift}\right) \left(\frac{1-e^{-\drift\boundarysep}}{1+e^{-\drift\boundarysep}}\right) \nonumber
\end{eqnarray}
{where $\text{sign}(x)$ denotes the sign function, which returns $+1$ if $x \geq 0$ and $-1$ if $x < 0$. In this context, it determines whether drift is positive (when accuracy $\geq 0.5$) or negative (when accuracy $< 0.5$).}

These equations provide point estimates. The bootstrap method extends this to uncertainty quantification.

\subsection{The EZ bootstrap procedure applied to simple diffusion}

The bootstrap exploits the known sampling distributions of the summary statistics. For a sample of size $n$, these distributions are:
\begin{eqnarray*}
n\acc &\sim& \text{Binomial}(n, p)\\
\meanrt &\sim& \mathcal{N}\left(m, \frac{s^2}{n}\right)\\
\varrt &\sim& \text{Gamma}\left(\frac{n-1}{2}, \frac{2s^2}{n-1}\right)
\end{eqnarray*}
where $p$, $m$, and $s^2$ are the true population values of accuracy, mean RT, and RT variance.

\subsection{Distributional assumptions and their justification} The binomial distribution for accuracy is exact. The normal distribution for mean RT is asymptotically valid by the central limit theorem; for finite samples, this approximation is accurate when $n \geq 30$ and the underlying RT distribution has finite variance. The Gamma distribution for variance is exact under the assumption that RTs are normally distributed; while RTs are not exactly normal, this approximation is reasonable for the EZ model when sample sizes are moderate ($n \geq 50$).
Note, however, that we do not make sample size recommendations based on these classical rules of thumb.  The procedure we propose is so fast that Monte Carlo studies are very feasible methods to evaluate the effect of violated assumptions.

\subsection{Independence assumption} We assume approximate independence between the three summary statistics. While these statistics are derived from the same data, they are asymptotically independent under the EZ model assumptions. For finite samples, correlations have negligible impact on coverage properties (see our simulation studies below). 

\subsection{Simulation studies}

We present two standard calibration studies to validate the EZ bootstrap method for the EZ diffusion model.  In both cases, the target of the study was the calibration of the uncertainty quantification in terms of credibility interval coverage -- that is, we sought to confirm that credibility intervals contain the true value at a rate commensurate to their nominal coverage probability (specifically, that 95\% credible intervals contain the true value 95\% of the time).

\subsubsection{Study 1: Single condition}
In Study 1, we evaluated coverage for the regular EZ diffusion model with a single condition. We simulated $n = 100$ trials from true parameters $\boundarysep = 1.0$, $\drift = 0.5$, and $\ndt = 0.2$. For each of 10,000 simulation repetitions, we computed credible intervals at the 95\% confidence level using the EZ bootstrap method (with $1000$ bootstrap samples). 

% \textit{Results.}
We evaluated coverage at the 95\% confidence level across all simulation repetitions. The EZ bootstrap method achieved excellent calibration. Coverage rates were 95.5\%, 94.8\%, and 96.0\% for boundary, drift, and non-decision time, respectively---all within expected sampling variation of the nominal 95\% level. When considering all three parameters simultaneously, the coverage rate was 87.7\%, closely matching the expected value ($0.95^3 \approx 0.857$).

\subsubsection{Study 2: Design matrix extension}
In Study 2, we evaluated coverage for the EZ diffusion model with a design matrix extension. We simulated data from four conditions using a design matrix with three regression weights for each parameter type (boundary, drift, and non-decision time), resulting in 9 beta weight parameters total. The true beta weights were: boundary weights $[1.0, 1.5, 2.0]$, drift weights $[0.4, 0.8, 1.2]$, and non-decision time weights $[0.2, 0.3, 0.4]$. {These parameter values were chosen to represent typical values in the EZ diffusion model \cite{matzke2009psychological}.}

The design matrices for boundary, drift, and non-decision time parameters were:
$$
X_\alpha = \begin{pmatrix}
1 & 0 & 0 \\
0 & 1 & 0 \\
0 & 0 & 1 \\
0 & 0 & 1
\end{pmatrix}\!\!,\,\,\,
X_\delta = \begin{pmatrix}
1 & 0 & 0 \\
0 & 1 & 0 \\
0 & 0 & 1 \\
0 & 1 & 0
\end{pmatrix}\!\!,\,\,\,
X_\tau = \begin{pmatrix}
1 & 0 & 0 \\
0 & 1 & 0 \\
0 & 0 & 1 \\
1 & 0 & 0
\end{pmatrix} 
$$
where each row corresponds to a condition and each column corresponds to a regression weight.
Each condition had $n = 100$ trials. We conducted 10,000 simulation repetitions, computing credible intervals at the 95\% confidence level using EZ bootstrap (with $1000$ bootstrap samples).

The EZ bootstrap method maintained excellent calibration even with the added complexity of the design matrix structure. For boundary weights, coverage rates were 95.7\%, 94.9\%, and 96.6\% across the three weights. For drift weights, coverage rates were 94.6\%, 94.2\%, and 98.0\%. For non-decision time weights, coverage rates were 96.2\%, 94.3\%, and 94.7\%. All coverage rates were within expected sampling variation of the nominal 95\% level.

\subsubsection{Computational performance} 
The EZ bootstrap method is exceptionally computationally efficient. The simulations were run on an off-the-shelf laptop, using a single core of an 11th Gen Intel\textregistered{} Core\texttrademark{} i7-1195G7 processor running at 2.90GHz.  For Study~1, the method required a mean of only 0.16~ms per estimation (SD = 0.02~ms), enabling rapid uncertainty quantification. For Study~2, with the added complexity of the design matrix, the method required a mean of 0.95~ms per estimation (SD = 0.13~ms). A comparable Bayesian analysis reported by \citeauthor{chávez_de_la_peña_vandekerckhove:in_press:hierarchical} (\citeyearNP{chávez_de_la_peña_vandekerckhove:in_press:hierarchical}, Appendix~D) took approximately 3,800~ms. The sub-millisecond computation times of the EZ bootstrap demonstrate that the method makes real-time\footnote{We have a strict definition of ``real-time'' as less than the duration of one display frame in a typical experimental setup -- usually at most 1/120th of a second or 8.33~ms.} uncertainty quantification feasible where traditional methods would be prohibitively slow.


\section{Application to the circular diffusion model}

The circular diffusion model \cite{Smith2016} extends the diffusion model to circular decision spaces, where evidence accumulation occurs in a two-dimensional disk rather than along a one-dimensional line. The EZ version of the circular diffusion model (EZ-CDM; \citeNP{QarehdaghiRad2024}) provides analytical expressions that map four summary statistics---circular mean of choice angles, circular variance of choice angles, mean response time, and response time variance---to four cognitive parameters: drift angle, drift magnitude, boundary radius, and non-decision time. This makes the EZ-CDM framework a synthetic likelihood model: it provides a likelihood function based on summary statistics rather than raw trial data.

\subsection{The EZ circular diffusion model}

The EZ circular diffusion model provides analytical expressions for four summary statistics from a set of $n$ circular decision trials.  Let $\MCA{}$ denote the observed circular mean of choice angles, $\VCA{}$ the observed circular variance of choice angles, $\MRT{}$ the observed mean response time, and $\VRT{}$ the observed response time variance. The model's four parameters are the drift angle $\driftangle$, drift magnitude $\driftmag$, boundary radius $\boundaryradius$, and non-decision time $\ndtcdm$.
This allows us to define forward equations 
$
f_\text{CDM}:
\left(\driftangle, \driftmag, \boundaryradius, \ndtcdm\right)
\mapsto
\left(\MCA{}, \VCA{}, \MRT{}, \VRT{}\right)
$ and inverse equations 
$
i_\text{CDM}:
\left(\MCA{}, \VCA{}, \MRT{}, \VRT{}\right)
\mapsto
\left(\driftangle, \driftmag, \boundaryradius, \ndtcdm\right)
$.

The forward equations $f_\text{CDM}: \left(\driftangle, \driftmag, \boundaryradius, \ndtcdm\right) \mapsto \left(\MCA{}, \VCA{}, \MRT{}, \VRT{}\right)$ map parameters to summary statistics. The choice-angle distribution is von Mises with mean $\mu = \driftangle$ and concentration $\kappa = \boundaryradius\driftmag$.
The circular mean and variance are then:
\begin{eqnarray}
\MCA{} &=& \driftangle \nonumber\\
\VCA{} &=& 1 - R \nonumber
\end{eqnarray}
where $R = {I_1(\kappa)}/{I_0(\kappa)}$ and $I_0$ and $I_1$ are modified Bessel functions of the first kind. The response time moments are:
\begin{eqnarray}
\MRT{} &=& \ndtcdm + \frac{\boundaryradius}{\driftmag} R \nonumber\\
\VRT{} &=& \frac{\boundaryradius^2}{\driftmag^2} R^2 + \frac{2\boundaryradius}{\driftmag^3} R - \boundaryradius^2 \nonumber
\end{eqnarray}

The inverse equations $i_\text{CDM}: \left(\MCA{}, \VCA{}, \MRT{}, \VRT{}\right) \mapsto \left(\driftangle, \driftmag, \boundaryradius, \ndtcdm\right)$ map summary statistics to parameters. The estimation proceeds in four steps:
\begin{eqnarray}
\hat{\driftangle} &=& \MCA{} \nonumber\\
R &=& 1 - \VCA{}, \quad \text{solve } \frac{I_1(\kappa)}{I_0(\kappa)} = R \text{ for } \kappa \nonumber\\
\hat{\driftmag}^2 &=& \frac{-\kappa^2 + \sqrt{\kappa^4 + 4\VRT{}(\kappa^2 R^2 + 2\kappa R)}}{2\VRT{}}, \quad \hat{\boundaryradius} = \frac{\kappa}{\hat{\driftmag}} \nonumber\\
\hat{\ndtcdm} &=& \MRT{} - \frac{\hat{\boundaryradius}}{\hat{\driftmag}}R \nonumber
\end{eqnarray}
where the solution for $\hat{\driftmag}^2$ is obtained by solving the quadratic equation derived from the forward equation for $\VRT{}$.
Solving $R = I_1(k)/I_0(k)$ for $k$ is faciliated by the piecewise estimator of \citeA[see pp.~88-93 for further detail]{Fisher1993}: $$k \approx \begin{cases} 2R + R^{3} + \tfrac{5}{6}R^{5}, & R < 0.53,\\ -0.4 + 1.39R + \dfrac{0.43}{1-R}, & 0.53 \le R < 0.85,\\ \dfrac{1}{R^{3} - 4R^{2} + 3R}, & R \ge 0.85.\end{cases}$$
As an excellent alternative, \citeA{QarehdaghiRad2024} suggest performing a single step of Newton-Raphson optimization. This approach typically provides higher accuracy than the piecewise approximation but requires extra computational effort.

% \textit{Sampling distributions.}
The sampling distributions of the four summary statistics determine the variability and robustness of EZ-CDM parameter estimates. Under the CDM, choice angles follow a von Mises distribution with concentration $\kappa = \boundaryradius\driftmag$, and response times follow the first-passage-time distribution of a two-dimensional Bessel process with drift.

For the circular mean of choice angles (MCA), the angular error $\widehat{\driftangle} - \driftangle$ is approximately normally distributed for large $N$ \cite{Fisher1993}, and so:
\begin{equation}
\widehat{\driftangle} \sim \mathcal{N}\left(\driftangle, \frac{1}{N\kappa R}\right). \nonumber
\end{equation}
The variance decreases rapidly with concentration $\kappa$, making estimation more difficult when drift magnitude or boundary radius are small.

For the circular variance ($\VCA{}$), the mean resultant length $L = \sqrt{\left(\frac{1}{N}\sum_{i=1}^{N} \cos(\theta_i)\right)^2 + \left(\frac{1}{N}\sum_{i=1}^{N} \sin(\theta_i)\right)^2}$ is approximately normal for moderate-to-large $N$, with mean $R$ and variance $\frac{1 - R^2}{N}$ \cite{Fisher1993}. Consequently, $\widehat{\VCA{}} = 1 - R$ is also asymptotically normal with the same variance. The circular variance is only weakly sensitive to outliers in angle (since angles wrap on the circle). However, parameter estimation becomes unstable when $R$ is near zero (because then the data are essentially uniform on the circle and provide limited information about the concentration parameter $\kappa$).

For mean response time ($\MRT{}$), the central limit theorem applies:
\begin{equation}
\widehat{\MRT{}} \sim \mathcal{N}\left(\MRT{}, \frac{\VRT{}}{N}\right) \nonumber
\end{equation}
Mean RT is robust for large $N$ but sensitive to RT contaminants.

For response time variance ($\VRT{}$), the sample variance is asymptotically normal but with slower convergence:
\begin{equation}
\widehat{\VRT{}} \sim \mathcal{N}\left(\VRT{}, \frac{m_4 - \VRT{}^2}{N}\right) \nonumber 
\end{equation}
where $m_4$ is the fourth central moment of the RT distribution. This statistic is the most contamination-sensitive in EZ-CDM, as a small number of extreme RTs can cause large upward shifts (see \citeNP{chávez_de_la_peña_etal:preprint:hierarchical}, and \citeNP{QarehdaghiRad2024}, for approaches using robust summary statistics).

The EZ bootstrap procedure can be applied to EZ-CDM by resampling these statistics from their asymptotic distributions. The angle-based statistics ($\MCA{}$ and $\VCA{}$) have well-behaved sampling distributions, while the RT-based statistics ($\MRT{}$ and $\VRT{}$) introduce most of the sampling variability in the inversion system, especially when solving for drift magnitude and boundary radius.

\subsection{Simulation studies}

We again present two calibration studies to validate the EZ bootstrap method for the EZ circular diffusion model.  The target of the study was again the calibration of the uncertainty quantification in terms of credibility interval coverage (95\% credibility).

\subsubsection{Study 1: Single condition}
In Study 1, we evaluated coverage for the regular EZ circular diffusion model with a single condition. We simulated $n = 100$ trials from true parameters $\driftangle = 0.0$, $\driftmag = 1.0$, $\boundaryradius = 1.0$, and $\ndtcdm = 0.2$. For each of 10,000 simulation repetitions, we computed credible intervals at the 95\% confidence level using the EZ bootstrap method (with $1000$ bootstrap samples). 

% \textit{Results.}
We evaluated coverage at the 95\% confidence level across 10,000 simulation repetitions. The EZ bootstrap method achieved excellent calibration. Coverage rates were 93.9\%, 94.3\%, 95.2\%, and 94.8\% for drift angle, drift magnitude, boundary radius, and non-decision time, respectively---all within expected sampling variation of the nominal 95\% level. When considering all four parameters simultaneously, the coverage rate was 82.4\%, closely matching the expected value ($0.95^4 \approx 0.815$).

\subsubsection{Study 2: Design matrix extension}
In Study 2, we evaluated coverage for the EZ circular diffusion model with a design matrix extension. We simulated data from four conditions using a design matrix with three regression weights for each parameter type (drift angle, drift magnitude, boundary radius, and non-decision time), resulting in 12 beta weight parameters total. The true beta weights were: drift angle weights $[0.0, 0.5, 1.0]$, drift magnitude weights $[0.5, 0.7, 0.9]$, boundary radius weights $[0.8, 1.0, 1.2]$, and non-decision time weights $[0.15, 0.2, 0.25]$ \cite{villarreal_etal:2024:graphical}.

The design matrices for drift angle, drift magnitude, boundary radius, and non-decision time parameters were:
$$
X_{\driftangle} = \begin{pmatrix}
1 & 0 & 0 \\
0 & 1 & 0 \\
0 & 0 & 1 \\
0 & 1 & 0
\end{pmatrix}\!\!,\,\,\,
X_{\driftmag} = \begin{pmatrix}
1 & 0 & 0 \\
0 & 1 & 0 \\
0 & 0 & 1 \\
0 & 1 & 0
\end{pmatrix}\!\!,$$
$$X_{\boundaryradius} = \begin{pmatrix}
1 & 0 & 0 \\
0 & 1 & 0 \\
0 & 0 & 1 \\
1 & 0 & 0
\end{pmatrix}\!\!,\,\,\,
X_{\ndtcdm} = \begin{pmatrix}
1 & 0 & 0 \\
0 & 1 & 0 \\
0 & 0 & 1 \\
0 & 0 & 1
\end{pmatrix} 
$$
where each row corresponds to a condition and each column corresponds to a regression weight.
Each condition had $n = 100$ trials. We conducted 10,000 simulation repetitions, computing credible intervals at the 95\% confidence level using EZ bootstrap (with $1000$ bootstrap samples).

The method maintained excellent calibration: For drift angle weights, coverage rates were 89.5\%, 94.5\%, and 94.2\% across the three weights. For drift magnitude weights, coverage rates were 94.5\%, 94.4\%, and 89.3\%. For boundary radius weights, coverage rates were 95.8\%, 94.8\%, and 93.0\%. For non-decision time weights, coverage rates were 95.8\%, 94.6\%, and 95.7\%. All coverage rates were within expected sampling variation of the nominal 95\% level.

\subsubsection{Computational performance}
For Study~1, the method required a mean of only 0.33~ms per estimation (SD = 0.03~ms), enabling rapid uncertainty quantification. For Study~2, with the added complexity of the design matrix, the method required a mean of 1.26~ms per estimation (SD = 0.19~ms). 

\section{Other models}

The EZ bootstrap method generalizes to any model in the EZ class -- that is, any model with analytical forward and inverse equations smoothly mapping parameters to summary statistics. Here we briefly describe one additional example: the one-high-threshold source-monitoring {multinomial processing tree (MPT)} model.

\subsection{One-high-threshold source-monitoring model}

The one-high-threshold source-monitoring MPT model \cite{BatchelderRiefer1990,BatchelderRiefer1999} is a multinomial processing tree model for source memory that allows participants to identify items as coming from Source~A, Source~B, or as new. The model has four parameters: detection probability $\Detect$, source discrimination probability $\Discrim$, criterion for new items $\Criterion$, and guessing probability $\Guess$.

The forward equations map parameters to observed response proportions. For old items from Source~A, the forward equations are:
\begin{eqnarray}
\AAstat &=& \Detect\Discrim + (1-\Detect)\Guess \nonumber\\
\BAstat &=& \Detect(1-\Discrim) + (1-\Detect)(1-\Guess) \nonumber
\end{eqnarray}
where $\AAstat = P(\text{``A''} \mid \text{old from A})$ and $\BAstat = P(\text{``B''} \mid \text{old from A})$. Analogous expressions hold for old items from Source~B:
\begin{eqnarray}
\ABstat &=& \Detect(1-\Discrim) + (1-\Detect)\Guess \nonumber\\
\BBstat &=& \Detect\Discrim + (1-\Detect)(1-\Guess) \nonumber
\end{eqnarray}
where $\ABstat = P(\text{``A''} \mid \text{old from B})$ and $\BBstat = P(\text{``B''} \mid \text{old from B})$. For new items, the model specifies:
\begin{eqnarray}
\NNstat &=& \Criterion \nonumber\\
\ANstat &=& (1-\Criterion)\Guess \nonumber\\
\BNstat &=& (1-\Criterion)(1-\Guess) \nonumber
\end{eqnarray}
where $\NNstat = P(\text{``new''} \mid \text{new})$, $\ANstat = P(\text{``A''} \mid \text{new})$, and $\BNstat = P(\text{``B''} \mid \text{new})$.
The inverse equations provide closed-form parameter estimates:
\begin{eqnarray}
\Criterion &=& \NNstat \nonumber\\
\Guess &=& \frac{\ANstat}{1-\NNstat} \nonumber\\
\Detect &=& \frac{\AAstat + \ABstat - 2\Guess}{1-2\Guess} \nonumber\\
\Discrim &=& \frac{\AAstat - (1-\Detect)\Guess}{\Detect} \nonumber
\end{eqnarray}
These linear relationships allow closed-form inversion, illustrating a simple but nontrivial MPT for which both forward and inverse transformations are available in closed form.

The sampling distributions for MPT models are multinomial: the observed counts in each response category follow a multinomial distribution with probabilities given by the forward equations. For old items from Source~A, the counts $(n_{\AAstat}, n_{\BAstat})$ follow:
\begin{equation}
(n_{\AAstat}, n_{\BAstat}) \sim \text{Multinomial}(n_A, (\AAstat, \BAstat)) \nonumber
\end{equation}
where $n_A$ is the number of old items from Source~A and $\AAstat + \BAstat = 1$. Similarly, for old items from Source~B:
\begin{equation}
(n_{\ABstat}, n_{\BBstat}) \sim \text{Multinomial}(n_B, (\ABstat, \BBstat)) \nonumber
\end{equation}
where $n_B$ is the number of old items from Source~B and $\ABstat + \BBstat = 1$. For new items:
\begin{equation}
(n_{\NNstat}, n_{\ANstat}, n_{\BNstat}) \sim \text{Multinomial}(n_N, (\NNstat, \ANstat, \BNstat)) \nonumber
\end{equation}
where $n_N$ is the number of new-item trials and $\NNstat + \ANstat + \BNstat = 1$. The EZ bootstrap procedure can be applied by resampling from these multinomial distributions and transforming through the inverse equations to obtain bootstrap parameter estimates.

We provide this solvable MPT example only as an illustration. The class of MPT models with tractable analytical inverses is rather narrow and does not reflect the complexity of most MPT applications, where estimation usually requires numerical optimization or Bayesian inference \cite{matzke_etal:2018:nonstandard}.

\section{Discussion}

The EZ bootstrap method provides fast uncertainty quantification. It maintains proper calibration while achieving dramatic speed improvements over previous methods such as the Bayesian implementation by \citeA{chávez_de_la_peña_vandekerckhove:in_press:hierarchical} -- the method achieves coverage rates that closely match nominal levels for both simple single-condition models and more complex design matrix extensions.

The method's success relies on the synthetic likelihood structure of EZ models. By operating directly on summary statistics with known sampling distributions, we avoid the computational cost of raw data resampling while preserving statistical validity. The transformation-of-variables approach generalizes to any model in the EZ class -- that is, any model with invertible mappings from parameters to summary statistics. The calibration results illustrate that our method maintains proper statistical properties while delivering computational efficiency.


\subsection{Limitations}

{The bootstrap method has a few limitations worth pointing out. First, the method is limited to the class of EZ models and their design matrix extensions. While identifiable models will typically \textit{have} both the forward and inverse equations, the equations are often not known or are mathematically inconvenient. Second, the method requires that summary statistics have known or approximately known sampling distributions. Approximations such as the normal distribution for mean RT and the Gamma for variance are asymptotically valid but may be less accurate for very small samples ($n < 30$) or when RT distributions are highly skewed, but good approximations of sampling distributions are known for very many summary statistics. Third, we assume approximate independence between summary statistics. While this holds asymptotically and is approximately valid for sufficiently large samples, correlations may be non-negligible for very small samples or models beyond our consideration.}


\subsection{Future directions}

Several extensions are possible. The method could be extended to yet more models with known mappings from parameters to summary statistics. Hierarchical extensions could enable multi-level modeling with summary statistics. Integration with experimental platforms could enable real-time semi-Bayesian design optimization and computerized adaptive tests.

Perhaps most promisingly, implicit mappings from parameters to summary statistics exist even when explicit equations are unknown. Techniques from statistical learning, such as normalizing flow neural networks, could be used to approximate these mappings for use in the synthetic bootstrap method, extending the approach to a broader class of models.


\section{Computational reproducibility}

All code associated with this project is available via GitHub. The simulation studies reported in this paper can be reproduced using the C implementation available in the GitHub repository \texttt{joachimvandekerckhove/ez-bootstrap}. The implementation requires the GNU Scientific Library (GSL) for random number generation and statistical distributions.

To reproduce the simulation studies, first
(1)~clone the repository, then
(2)~install dependencies if needed (e.g., install \texttt{libgsl-dev} to obtain GSL development libraries on Debian/Ubuntu systems), and 
(3)~run the simulation script
\texttt{./run\_all.sh}.

The script executes both simulation studies for the EZ-DDM and EZ-CDM with the following editable parameters: 10,000 simulation repetitions, sample size $n = 100$ trials per condition, and 1,000 bootstrap samples per repetition. The simulation outputs include coverage rates for each parameter (or beta weight) and computational performance metrics.

{A Python implementation of the EZ-DDM is also available as part of the larger EZAS package (\url{https://github.com/joachimvandekerckhove/ezas}), which uses the same method but is not as fast as the C implementation.}


\ifdefined\apaformat

\newcommand{\dec}[2]{\noindent\textbf{#1} --- #2}

\section{Declarations}

\dec{Funding}{This work was supported by the National Research Council.}
   
\dec{Competing interests}{The authors report no competing interest.}

\dec{Ethics approval}{N/A}

\dec{Consent to participate}{N/A}

\dec{Consent for publication}{N/A}

\dec{Availability of data and materials}{N/A}

\dec{Code availability}{See Section ``Computational reproducibility.''}

\fi


\bibliography{includes/references,includes/cidlab}\null

\subsection{Disclaimer}
The views expressed are those of the authors and do not necessarily reflect the official policy or position of the Department of the Air Force, the Department of Defense, or the U.S. Government. 


% \appendix
% \ifdefined\apaformat
% \section{Reproducing the simulation studies}
% \else
% \section{Appendix: Reproducing the simulation studies}
% \fi

\end{document}
"
% To compile in original format: pdflatex main.tex

% \def\apaformat{1}

\ifdefined\apaformat
  % APA-style preamble for draft format
% This file is included by main.tex when compiling in APA format

\documentclass[man,12pt]{apa6}

\usepackage{
  amsmath,amssymb,amsfonts,
  csquotes,
  epsfig,epstopdf,epigraph,
  fancyhdr,
  graphicx,
  hyperref,
  mathtools,
  setspace,
  subfig,
  tikz,
  url,
  apacite}

% Set up bibliography style for APA format
\bibliographystyle{apacite}

% Redefine cidlab-specific commands to work with apa6
\newcommand{\leadauthor}[1]{}
\newcommand{\significancestatement}[1]{}
\newcommand{\authorcontributions}[1]{}
\newcommand{\authordeclaration}[1]{}
\newcommand{\equalauthors}[1]{}
\newcommand{\correspondingauthor}[1]{}
\newcommand{\dates}[1]{}
\newcommand{\verticaladjustment}[1]{}
\newcommand{\firststyle}{fancy}
\newcommand{\abscontentformatted}{}
\newcommand{\abscontent}{}
\newcommand{\nocaption}{}

% Redefine section commands for APA format
\let\origsection\section
\let\origsubsection\subsection
\renewcommand{\section}[1]{\origsection{#1}}
\renewcommand{\subsection}[1]{\origsubsection{#1}}

% Redefine citation commands for APA format
% \citeNP is already defined by apacite

% Define introduction command for APA format
\newcommand{\introduction}[1]{\section{Introduction}\label{sec:introduction}}


% Include shared metadata
% Shared metadata for both document formats
% This file is included by both draft.tex and pretty.tex

% Title, authors and affiliations - format-specific
\ifdefined\apaformat
  \title{Fast Uncertainty Quantification in EZ Cognitive Models}
  % APA-specific formatting
  \shorttitle{EZ Cognitive Models}

  % APA format uses combined author/affiliation
  \author{Joachim Vandekerckhove$^{a,\dagger}$ and Elizabeth L.\ Fox$^{b}$\vspace{3ex}}
  \affiliation{%
  $^a$ University of California, Irvine\\
  $^b$ Air Force Research Laboratory\\\vspace{3ex}
  $^\dagger$ To whom correspondence should be addressed. E-mail: joachim@uci.edu.}
\else
  % cidlab format uses individual author/affiliation with symbols
  \title{Fast uncertainty quantification in EZ cognitive models}

  \author[a,*]{Joachim Vandekerckhove}
  \author[b]{Elizabeth L.\ Fox}

  \affil[a]{University of California, Irvine}
  \affil[b]{Air Force Research Laboratory}

  \leadauthor{Vandekerckhove} 
    
   \authorcontributions{All authors contributed to the final draft.}
   \authordeclaration{This work was supported by a National Research Council Fellowship to JV.}
   \correspondingauthor{\textsuperscript{*}Correspondence concerning this article should be addressed to Joachim Vandekerckhove (joachim@uci.edu).}
   \equalauthors{}
    
   \dates{Draft of \today{}.}%  Planned outlet: \textit{Advances in Methods and Practices in Psychological Science}.}
    
\fi

% Abstract - format-specific
\ifdefined\apaformat
  % APA format uses \abstract command
  \abstract{%
    

Classical approaches to uncertainty quantification in cognitive modeling rely on computationally expensive Monte Carlo methods or resampling of raw trial data, which creates a barrier for real-time analysis and large-scale studies. We present a computationally efficient bootstrap method that operates directly on summary statistics, exploiting the synthetic likelihood structure of a small class of cognitive models that includes the simple diffusion model and the circular diffusion model in addition to signal detection and multinomial processing trees. The method relies on a numerical transformation-of-variables technique in which known sampling distributions of summary statistics of behavior are parametrically bootstrapped, after which the resampled statistics are transformed to parameter estimates with a known analytical system. This approach does not require additional assumptions beyond those already made by the models themselves, but achieves over 1000-fold speed improvements over already efficient fully Bayesian methods. The proposed method makes real-time uncertainty quantification accessible and enables new applications in adaptive testing, meta-analyses, and exploratory data analysis.
  }
\fi

% Keywords
\keywords{cognitive modeling, diffusion model, bootstrap, uncertainty quantification, EZ diffusion}

\graphicspath{{./figures/}}

\let\citep\cite
\let\citet\citeA


\else
  % Original cidlab-style preamble for pretty format
% This file is included by main.tex when compiling in original format

\documentclass[9pt,twocolumn,twoside]{cidlab/cidlab}

\usepackage{
  amsmath,amssymb,amsfonts,
  csquotes,
  epsfig,epstopdf,epigraph,
  graphicx,
  mathtools,
  subfig,
  tikz,
  url}
\usepackage[super]{nth}
\usepackage[english]{babel}

% Define introduction command that does nothing in cidlab format
\newcommand{\introduction}[1]{}


% Abstract for cidlab format
\begin{abstract} 


Classical approaches to uncertainty quantification in cognitive modeling rely on computationally expensive Monte Carlo methods or resampling of raw trial data, which creates a barrier for real-time analysis and large-scale studies. We present a computationally efficient bootstrap method that operates directly on summary statistics, exploiting the synthetic likelihood structure of a small class of cognitive models that includes the simple diffusion model and the circular diffusion model in addition to signal detection and multinomial processing trees. The method relies on a numerical transformation-of-variables technique in which known sampling distributions of summary statistics of behavior are parametrically bootstrapped, after which the resampled statistics are transformed to parameter estimates with a known analytical system. This approach does not require additional assumptions beyond those already made by the models themselves, but achieves over 1000-fold speed improvements over already efficient fully Bayesian methods. The proposed method makes real-time uncertainty quantification accessible and enables new applications in adaptive testing, meta-analyses, and exploratory data analysis.
\end{abstract}

\significancestatement{}
\authorcontributions{}
\authordeclaration{}
\equalauthors{}
\correspondingauthor{\textsuperscript{$\dagger$}To whom correspondence should be addressed. E-mail: joachim@uci.edu.}

\dates{Draft of \today.}
% \dates{Planned outlet: \textit{Nature Human Behaviour}.}

% Include shared metadata
% Shared metadata for both document formats
% This file is included by both draft.tex and pretty.tex

% Title, authors and affiliations - format-specific
\ifdefined\apaformat
  \title{Fast Uncertainty Quantification in EZ Cognitive Models}
  % APA-specific formatting
  \shorttitle{EZ Cognitive Models}

  % APA format uses combined author/affiliation
  \author{Joachim Vandekerckhove$^{a,\dagger}$ and Elizabeth L.\ Fox$^{b}$\vspace{3ex}}
  \affiliation{%
  $^a$ University of California, Irvine\\
  $^b$ Air Force Research Laboratory\\\vspace{3ex}
  $^\dagger$ To whom correspondence should be addressed. E-mail: joachim@uci.edu.}
\else
  % cidlab format uses individual author/affiliation with symbols
  \title{Fast uncertainty quantification in EZ cognitive models}

  \author[a,*]{Joachim Vandekerckhove}
  \author[b]{Elizabeth L.\ Fox}

  \affil[a]{University of California, Irvine}
  \affil[b]{Air Force Research Laboratory}

  \leadauthor{Vandekerckhove} 
    
   \authorcontributions{All authors contributed to the final draft.}
   \authordeclaration{This work was supported by a National Research Council Fellowship to JV.}
   \correspondingauthor{\textsuperscript{*}Correspondence concerning this article should be addressed to Joachim Vandekerckhove (joachim@uci.edu).}
   \equalauthors{}
    
   \dates{Draft of \today{}.}%  Planned outlet: \textit{Advances in Methods and Practices in Psychological Science}.}
    
\fi

% Abstract - format-specific
\ifdefined\apaformat
  % APA format uses \abstract command
  \abstract{%
    

Classical approaches to uncertainty quantification in cognitive modeling rely on computationally expensive Monte Carlo methods or resampling of raw trial data, which creates a barrier for real-time analysis and large-scale studies. We present a computationally efficient bootstrap method that operates directly on summary statistics, exploiting the synthetic likelihood structure of a small class of cognitive models that includes the simple diffusion model and the circular diffusion model in addition to signal detection and multinomial processing trees. The method relies on a numerical transformation-of-variables technique in which known sampling distributions of summary statistics of behavior are parametrically bootstrapped, after which the resampled statistics are transformed to parameter estimates with a known analytical system. This approach does not require additional assumptions beyond those already made by the models themselves, but achieves over 1000-fold speed improvements over already efficient fully Bayesian methods. The proposed method makes real-time uncertainty quantification accessible and enables new applications in adaptive testing, meta-analyses, and exploratory data analysis.
  }
\fi

% Keywords
\keywords{cognitive modeling, diffusion model, bootstrap, uncertainty quantification, EZ diffusion}

\graphicspath{{./figures/}}

\let\citep\cite
\let\citet\citeA


\AtBeginDocument{
\verticaladjustment{-2pt}
\thispagestyle{firststyle}
\ifthenelse{\boolean{shortarticle}}{\ifthenelse{\boolean{singlecolumn}}{\abscontentformatted}{\abscontent}}{}
}

\fi
% Main document content starts here


\setlength {\marginparwidth }{2cm}


\usepackage{todonotes}

\newif\ifshownotes

\newcommand{\definenote}[2]{%
  \expandafter\newcommand\csname #1\endcsname[1]{%
    \ifshownotes
      \todo[inline,color=#2]{##1 -\textit{#1}}%
    \fi
  }%
}

\shownotesfalse

% https://en.wikipedia.org/wiki/X11_color_names
\definenote{jv}{Lime}
\definenote{elf}{Gold}


% Comment this out to remove notes:
\shownotestrue

% Macros for EZ model summary statistics and parameters
% This file contains all custom macros used throughout the document

% Signal Detection Theory (SDT) parameters
\newcommand{\dprime}{d'}
\newcommand{\criterion}{c}

% Signal Detection Theory (SDT) summary statistics
\newcommand{\HitRate}{H}
\newcommand{\FalseAlarm}{F}

% EZ diffusion model parameters
\newcommand{\drift}{\delta}
\newcommand{\boundarysep}{\alpha}
\newcommand{\ndt}{\tau}

% EZ diffusion model summary statistics
\newcommand{\acc}{\hat{P}}
\newcommand{\meanrt}{\hat{M}}
\newcommand{\varrt}{\hat{V}}

% EZ circular diffusion model (EZ-CDM) parameters
\newcommand{\driftangle}{\theta_v}
\newcommand{\driftmag}{v}
\newcommand{\boundaryradius}{a}
\newcommand{\ndtcdm}{t_0}

% EZ circular diffusion model (EZ-CDM) summary statistics
\newcommand{\MCA}{M_a}
\newcommand{\VCA}{V_a}
\newcommand{\MRT}{M_t}
\newcommand{\VRT}{V_t}

% One-high-threshold source-monitoring MPT model parameters
\newcommand{\Detect}{D}
\newcommand{\Discrim}{d}
\newcommand{\Criterion}{c}
\newcommand{\Guess}{g}

% One-high-threshold source-monitoring MPT model summary statistics
\newcommand{\AAstat}{A_A}
\newcommand{\BAstat}{B_A}
\newcommand{\ABstat}{A_B}
\newcommand{\BBstat}{B_B}
\newcommand{\NNstat}{N_N}
\newcommand{\ANstat}{A_N}
\newcommand{\BNstat}{B_N}


\usepackage{algorithm}
\usepackage{algpseudocode}
\usepackage{listings}
\lstset{extendedchars=true,upquote=true}
% Handle UTF-8 characters in listings using literate option
\lstdefinestyle{utf8style}{
  extendedchars=true,
  upquote=true,
  literate={≈}{{$\approx$}}1
}
\lstset{style=utf8style}

\renewcommand{\subsubsection}[1]{\textit{#1}.}

\begin{document}

\maketitle

\introduction{}

\noindent{}%
Cognitive models serve dual purposes in psychological research. On one hand, they can be used to understand the cognitive infrastructure---the underlying mechanisms and processes that give rise to behavior. On the other hand, cognitive models can function as measurement tools, translating behavioral observations into interpretable psychological parameters. This paper focuses on the latter: we develop computationally efficient methods for uncertainty quantification in cognitive models used as measurement instruments. The need for computational efficiency becomes critical in large-scale applications such as meta-analyses across hundreds of studies, real-time experimental analysis during data collection, and individual differences research requiring parameter estimation for thousands of participants. Traditional approaches to uncertainty quantification, which rely on computationally expensive resampling or Bayesian inference, create barriers in these contexts. We propose a bootstrap method that exploits the synthetic likelihood structure of a class of cognitive models we call `EZ' models, achieving dramatic speed improvements while maintaining statistical validity.

\section{Synthetic likelihood methods}

Synthetic likelihood methods provide a framework for statistical inference when the likelihood function is intractable but summary statistics with known or estimable distributions are available \cite{wood_statistical_2010,Price2017}. The core idea is to replace the full likelihood with a likelihood based on summary statistics, which can be computed efficiently even when the full data likelihood cannot.

The synthetic likelihood approach was introduced by \citeauthor{wood_statistical_2010} for inference in complex ecological systems with chaotic dynamics. When direct likelihood evaluation is impossible, one can extract phase-insensitive summary statistics from the data and compare them to summary statistics simulated from the model. This approach has been extended to Bayesian settings \cite{Price2017}, where the synthetic likelihood is used within a Bayesian framework to enable posterior inference.

Certain cognitive models, which we will refer to here as the `EZ' models (after the EZ diffusion model; \citeNP{WagenmakersEtAl2007}), similarly fit the description of synthetic likelihood models, and provide a natural application of methods specific to that class. For example, in the EZ diffusion model, rather than working with the full likelihood over all trial-level data, we work with three summary statistics: accuracy rate, mean response time, and response time variance. These statistics have known sampling distributions, allowing us to construct a likelihood function directly from the summaries \cite{chávez_de_la_peña_vandekerckhove:in_press:hierarchical}. This synthetic likelihood approach is particularly powerful because the EZ model provides analytical expressions that map parameters to these summary statistics, eliminating the need for simulation, and paving the way for uncertainty quantification using a variation on a classic transformation-of-variables technique, which we will describe after introducing the model class.

\section{The class of EZ models}

We will rely on a class of cognitive models we call `EZ'.  EZ models are defined by the existence of an invertible system of equations that maps a model parameter vector $\theta$ to a set of summary statistics $S$.  The vector $\theta$ need not contain \textit{all} of the model's parameters, but may be limited to a subset of interest (as in the application by \citeNP{wood_statistical_2010}).  The system consists of a set of smooth `forward' equations $f$ and `inverse' equations $i$:
\begin{eqnarray}
    f& :&  \theta \mapsto S \nonumber\\
    i& :&  S \mapsto \theta\label{eq:generic:inv}
\end{eqnarray}

The EZ class contains some well-known cognitive models.  We take signal detection theory (SDT; \citeNP{swets_green:1966:signal}) as an example:
\begin{eqnarray}
    (\HitRate,\FalseAlarm) &=& f_\text{SDT}(\dprime,\criterion) \nonumber \\
    &\Leftrightarrow& \left\{
    \begin{array}{rcl}
         \HitRate&=& \Phi(\dprime - \criterion) \\
         \FalseAlarm&=& \Phi(-\criterion) \\
    \end{array}\right.
\end{eqnarray}
where $\Phi$ denotes the standard normal cumulative distribution function. The inverse equations map summary statistics back to parameters:
\begin{eqnarray}
    (\dprime,\criterion) &=& i_\text{SDT}(\HitRate,\FalseAlarm) \nonumber \\
    &\Leftrightarrow& \left\{
    \begin{array}{rcl}
         \dprime&=& \Phi^{-1}(\HitRate) - \Phi^{-1}(\FalseAlarm) \\
         \criterion&=& -\Phi^{-1}(\FalseAlarm) \\
    \end{array}\right.
\end{eqnarray}
where $\Phi^{-1}$ denotes the inverse standard normal cumulative distribution function. Here, $\HitRate$ denotes the hit rate (probability of correctly identifying a signal), $\FalseAlarm$ denotes the false alarm rate (probability of incorrectly identifying a signal when none is present), $\dprime$ denotes sensitivity (the ability to discriminate signal from noise), and $\criterion$ denotes the criterion (the decision threshold).

Other members of the EZ class include the EZ diffusion model (\citeNP{WagenmakersEtAl2007}), the EZ circular diffusion model \cite{QarehdaghiRad2024}, and some simple multinomial processing tree models with analytical solutions. {These models are discussed in detail below.}


\section{The EZ bootstrap procedure}\label{sec:ez_bootstrap}

We propose a bootstrap method we call `EZ bootstrap' that exploits the synthetic likelihood structure of EZ models. The method operates directly on summary statistics, using their known sampling distributions to perform a parametric bootstrap. 

In the abstract, EZ bootstrap proceeds along the following steps. First, (1) compute summary statistics $S$ from the {observed} data. Then, (2) for each bootstrap repetition $b = 1, \ldots, B$: (2a) sample summary statistics $S^*_b$ from their known sampling distributions $\mathcal{D}_S$ (the requirement that these distributions be known or estimable is fundamental to the method), and (2b) transform $S^*_b$ through the inverse equations $i$ to obtain bootstrap parameter estimates $\theta^*_b = i(S^*_b)$. Finally, (3) aggregate the bootstrap parameter estimates $\{\theta^*_1, \ldots, \theta^*_B\}$ to compute means, standard deviations, and credible intervals.

This procedure is formalized in Algorithm~\ref{alg:ez_bootstrap}. EZ bootstrap is fundamentally a transformation-of-variables approach to uncertainty estimation. Rather than resampling raw data and re-estimating parameters, we resample the summary statistics from their known distributions, then transform these resampled statistics through the inverse equations to obtain parameter estimates. {Specifically, in Algorithm~\ref{alg:ez_bootstrap} we sample $S^*_b$ from $\mathcal{D}_S$ for each bootstrap repetition $b = 1, \ldots, B$, then compute $\theta^*_b = i(S^*_b)$ to obtain the bootstrap parameter estimates $\{\theta^*_1, \ldots, \theta^*_B\}$.} This transformation preserves the statistical properties of the bootstrap while eliminating the computational cost of data resampling. 

\subsection{Transformation of variables for uncertainty quantification}
Transformation of variables is a fundamental method in statistics for propagating uncertainty through deterministic transformations (\citeNP{rice2006mathematical}, see also \citeNP{hayes2013relative}). When a random variable $X$ with known distribution is transformed through a function $Y = g(X)$, the distribution of $Y$ can be derived using the change-of-variables formula. This technique is central to the EZ bootstrap method.

In the context of uncertainty quantification, transformation of variables allows us to derive the distribution of parameter estimates from the distribution of summary statistics in EZ models. The inverse equations (Eq.~\ref{eq:generic:inv}) provide a deterministic transformation from summary statistics to parameters. By applying this transformation to bootstrap samples of the summary statistics, we obtain bootstrap samples of the parameters, which can then be used to quantify uncertainty.

The key advantage of this approach is computational efficiency. Rather than resampling raw data and re-estimating parameters for each bootstrap sample---which would require expensive optimization or simulation \cite{efron1994introduction}---we resample the summary statistics from their known distributions and apply the analytical transformation, {as formalized in Algorithm~\ref{alg:ez_bootstrap}}. This eliminates the computational bottleneck while preserving the statistical properties of the bootstrap.

The transformation-of-variables approach is made possible in EZ models particularly because the inverse transformation is analytical, smooth, and one-to-one (within the parameter space, or a relevant subspace thereof). 

\begin{algorithm}[bt]
\caption{EZ Bootstrap Procedure}
\label{alg:ez_bootstrap}
\begin{algorithmic}[1]
\Require Observed summary statistics $S$, number of bootstrap samples $B$, inverse transformation $i: S \mapsto \theta$, sampling distributions $\mathcal{D}_S$ for each component of $S$
\Ensure Bootstrap parameter estimates $\{\theta^*_1, \ldots, \theta^*_B\}$
\State $\Theta^* \leftarrow \emptyset$
\For{$b = 1$ to $B$}
    \State Sample $S^*_b \sim \mathcal{D}_S$ from known sampling distributions of $S$
    \State $\theta^*_b \leftarrow i(S^*_b)$
    \State $\Theta^* \leftarrow \Theta^* \cup \{\theta^*_b\}$
\EndFor
\Return $\Theta^*$
\end{algorithmic}
\end{algorithm}

\subsection{Theoretical justification for coverage preservation} The bootstrap procedure maintains proper coverage properties when the inverse transformation $i: S \mapsto \theta$
is a continuous, differentiable function. Under the bootstrap sampling scheme, we generate $S^*$ from its known distribution, then compute $\theta^* = i(S^*)$. Since $i$ is deterministic and the bootstrap samples $S^*$ have the correct sampling distributions, the transformed samples $\theta^*$ correctly represent the sampling distribution of the parameter estimates. This follows from the invariance property of bootstrap methods under smooth transformations \cite{rice2006mathematical}. The coverage properties are preserved because the bootstrap distribution of the transformed parameters matches the true sampling distribution of the parameter estimates -- up to the accuracy of the assumed distributions for the summary statistics.

\section{Design matrix extension}

In applied contexts, we often constrain parameters across conditions. These constraints can be expressed using design matrices. For example, {for diffusion models,} if drift rates, {$\drift$,} vary linearly with a stimulus property $x_i$, we can write:
$$\drift_i = \beta_0 + \beta_1 x_i.$$
If they vary quarilinearly, then:
$$\drift_i = \beta_0 + \beta_1 x_i + \beta_2 x_i^2.$$

This generalizes, so that parameter constraints can be expressed as:
$$\Delta = X \times \boldsymbol{\beta}$$
where $\Delta$ is a vector of parameters, $X$ is a design matrix, and $\boldsymbol{\beta}$ is a vector of regression weights.

If the model parameters $\Delta$ are known and the model is identified, the weights $\boldsymbol{\beta}$ can be derived via matrix inversion:
\begin{equation}
    \hat{\boldsymbol{\beta}} = X^{-1} \times \Delta \label{eq:design:solution}
\end{equation}

Importantly, adding the design matrix extension \textit{keeps the model inside the EZ class} and does not change the bootstrap logic.  This is easy to prove: Let $i: S \mapsto \theta$ denote the EZ inverse transformation mapping summary statistics $S$ to parameters $\theta$ (Eq.~\ref{eq:generic:inv}), and let $h: \theta \mapsto \boldsymbol{\beta}$ denote the design matrix transformation mapping parameters $\theta$ to beta weights $\boldsymbol{\beta}$ via $h(\theta) = X^{-1}\theta$ (where $X$ is the design matrix). The basic EZ bootstrap generates bootstrap samples $\theta^* = i(S^*)$ where $S^*$ is resampled from the known distribution of summary statistics. With the design matrix extension, we generate bootstrap samples $\boldsymbol{\beta}^* = h(i(S^*)) = X^{-1}i(S^*)$. 
Since $i$ is a deterministic, smooth transformation (by the definition of EZ models) and $h$ is a deterministic linear transformation (matrix multiplication), their composition $\left(h \circ i\right): S \mapsto \boldsymbol{\beta}$ is also deterministic and smooth. By the same invariance property of bootstrap methods under smooth transformations \cite{rice2006mathematical}, if the bootstrap distribution of $S^*$ correctly represents the sampling distribution of $S$, then the bootstrap distribution of $\boldsymbol{\beta}^* = (h \circ i)(S^*)$ correctly represents the sampling distribution of $\boldsymbol{\beta} = (h \circ i)(S)$. Therefore, the design matrix extension preserves the bootstrap's coverage properties: the model remains in the EZ class because we still use the EZ inverse transformation, and the bootstrap logic is unchanged because we are still propagating uncertainty through a deterministic transformation.

\section{Application to the simple diffusion model}

The drift diffusion model (DDM) provides a unified account of decision-making processes, modeling both response accuracy and response time as the result of an evidence accumulation process \cite{Ratcliff1978,Wagenmakers2009Bertelson}. This framework has proven valuable in cognitive psychometrics, enabling the translation of behavioral patterns into interpretable cognitive parameters \cite<e.g.,>{pe_etal:2013:relationship,schubert_etal:2019:neuroscience,vandekerckhove:2014:simultaneous}.

The EZ diffusion model \cite{WagenmakersEtAl2007} was originally proposed as a method to simplify use of the DDM by providing analytical expressions that map three summary statistics---accuracy rate, mean response time, and response time variance---to three cognitive parameters: drift rate, boundary separation, and non-decision time. This makes the EZ framework a synthetic likelihood model: it provides a likelihood function based on summary statistics rather than raw trial data.

\subsection{The EZ diffusion model}

The EZ diffusion model provides analytical expressions for three summary statistics from a set of $n$ two-choice decision trials.  Let $\acc$ denote the observed accuracy rate, $\meanrt$ the observed mean response time, and $\varrt$ the observed response time variance. The model's three parameters are the drift rate $\drift$, boundary separation $\boundarysep$, and non-decision time $\ndt$.
This allows us to define forward equations 
$
f_\text{DIF}:
\left(\drift,\boundarysep,\ndt\right)
\mapsto
\left(\acc, \meanrt, \varrt\right)
$ and inverse equations 
$
i_\text{DIF}:
\left(\acc, \meanrt, \varrt\right)
\mapsto
\left(\drift,\boundarysep,\ndt\right)
$.

The EZ forward equations map parameters to summary statistics. Let $y = \exp(-\boundarysep \drift)$ where $\boundarysep$ is boundary separation and $\drift$ is drift rate. Then,
\begin{eqnarray}
\acc &=& \frac{1}{1+y} \nonumber\\
\meanrt &=& \ndt + \left(\frac{\boundarysep}{2\drift}\right) \left(\frac{1-y}{1+y}\right) \nonumber\\
\varrt &=& \left(\frac{\boundarysep}{2\drift^3}\right) \left(\frac{1-2\boundarysep\drift y - y^2}{(1+y)^2}\right) \nonumber
\end{eqnarray}
When $\drift = 0$ (drift rate is zero), the forward equations simplify to $\acc = 0.5$, $\meanrt = \ndt + \boundarysep$, and $\varrt = \boundarysep^2$.

The EZ inverse equations map summary statistics to parameters. Let $y = \log(\acc/(1-\acc))$ be the logit of accuracy. Then,
\begin{eqnarray}
\drift &=& \text{sign}\left(\acc - \frac{1}{2}\right) \sqrt[4]{\frac{y(\acc^2 y - \acc y + \acc - 1/2)}{\varrt}} \nonumber\\
\boundarysep &=& \frac{y}{\drift} \nonumber\\
\ndt &=& \meanrt - \left(\frac{\boundarysep}{2\drift}\right) \left(\frac{1-e^{-\drift\boundarysep}}{1+e^{-\drift\boundarysep}}\right) \nonumber
\end{eqnarray}
{where $\text{sign}(x)$ denotes the sign function, which returns $+1$ if $x \geq 0$ and $-1$ if $x < 0$. In this context, it determines whether drift is positive (when accuracy $\geq 0.5$) or negative (when accuracy $< 0.5$).}

These equations provide point estimates. The bootstrap method extends this to uncertainty quantification.

\subsection{The EZ bootstrap procedure applied to simple diffusion}

The bootstrap exploits the known sampling distributions of the summary statistics. For a sample of size $n$, these distributions are:
\begin{eqnarray*}
n\acc &\sim& \text{Binomial}(n, p)\\
\meanrt &\sim& \mathcal{N}\left(m, \frac{s^2}{n}\right)\\
\varrt &\sim& \text{Gamma}\left(\frac{n-1}{2}, \frac{2s^2}{n-1}\right)
\end{eqnarray*}
where $p$, $m$, and $s^2$ are the true population values of accuracy, mean RT, and RT variance.

\subsection{Distributional assumptions and their justification} The binomial distribution for accuracy is exact. The normal distribution for mean RT is asymptotically valid by the central limit theorem; for finite samples, this approximation is accurate when $n \geq 30$ and the underlying RT distribution has finite variance. The Gamma distribution for variance is exact under the assumption that RTs are normally distributed; while RTs are not exactly normal, this approximation is reasonable for the EZ model when sample sizes are moderate ($n \geq 50$).
Note, however, that we do not make sample size recommendations based on these classical rules of thumb.  The procedure we propose is so fast that Monte Carlo studies are very feasible methods to evaluate the effect of violated assumptions.

\subsection{Independence assumption} We assume approximate independence between the three summary statistics. While these statistics are derived from the same data, they are asymptotically independent under the EZ model assumptions. For finite samples, correlations have negligible impact on coverage properties (see our simulation studies below). 

\subsection{Simulation studies}

We present two standard calibration studies to validate the EZ bootstrap method for the EZ diffusion model.  In both cases, the target of the study was the calibration of the uncertainty quantification in terms of credibility interval coverage -- that is, we sought to confirm that credibility intervals contain the true value at a rate commensurate to their nominal coverage probability (specifically, that 95\% credible intervals contain the true value 95\% of the time).

\subsubsection{Study 1: Single condition}
In Study 1, we evaluated coverage for the regular EZ diffusion model with a single condition. We simulated $n = 100$ trials from true parameters $\boundarysep = 1.0$, $\drift = 0.5$, and $\ndt = 0.2$. For each of 10,000 simulation repetitions, we computed credible intervals at the 95\% confidence level using the EZ bootstrap method (with $1000$ bootstrap samples). 

% \textit{Results.}
We evaluated coverage at the 95\% confidence level across all simulation repetitions. The EZ bootstrap method achieved excellent calibration. Coverage rates were 95.5\%, 94.8\%, and 96.0\% for boundary, drift, and non-decision time, respectively---all within expected sampling variation of the nominal 95\% level. When considering all three parameters simultaneously, the coverage rate was 87.7\%, closely matching the expected value ($0.95^3 \approx 0.857$).

\subsubsection{Study 2: Design matrix extension}
In Study 2, we evaluated coverage for the EZ diffusion model with a design matrix extension. We simulated data from four conditions using a design matrix with three regression weights for each parameter type (boundary, drift, and non-decision time), resulting in 9 beta weight parameters total. The true beta weights were: boundary weights $[1.0, 1.5, 2.0]$, drift weights $[0.4, 0.8, 1.2]$, and non-decision time weights $[0.2, 0.3, 0.4]$. {These parameter values were chosen to represent typical values in the EZ diffusion model \cite{matzke2009psychological}.}

The design matrices for boundary, drift, and non-decision time parameters were:
$$
X_\alpha = \begin{pmatrix}
1 & 0 & 0 \\
0 & 1 & 0 \\
0 & 0 & 1 \\
0 & 0 & 1
\end{pmatrix}\!\!,\,\,\,
X_\delta = \begin{pmatrix}
1 & 0 & 0 \\
0 & 1 & 0 \\
0 & 0 & 1 \\
0 & 1 & 0
\end{pmatrix}\!\!,\,\,\,
X_\tau = \begin{pmatrix}
1 & 0 & 0 \\
0 & 1 & 0 \\
0 & 0 & 1 \\
1 & 0 & 0
\end{pmatrix} 
$$
where each row corresponds to a condition and each column corresponds to a regression weight.
Each condition had $n = 100$ trials. We conducted 10,000 simulation repetitions, computing credible intervals at the 95\% confidence level using EZ bootstrap (with $1000$ bootstrap samples).

The EZ bootstrap method maintained excellent calibration even with the added complexity of the design matrix structure. For boundary weights, coverage rates were 95.7\%, 94.9\%, and 96.6\% across the three weights. For drift weights, coverage rates were 94.6\%, 94.2\%, and 98.0\%. For non-decision time weights, coverage rates were 96.2\%, 94.3\%, and 94.7\%. All coverage rates were within expected sampling variation of the nominal 95\% level.

\subsubsection{Computational performance} 
The EZ bootstrap method is exceptionally computationally efficient. The simulations were run on an off-the-shelf laptop, using a single core of an 11th Gen Intel\textregistered{} Core\texttrademark{} i7-1195G7 processor running at 2.90GHz.  For Study~1, the method required a mean of only 0.16~ms per estimation (SD = 0.02~ms), enabling rapid uncertainty quantification. For Study~2, with the added complexity of the design matrix, the method required a mean of 0.95~ms per estimation (SD = 0.13~ms). A comparable Bayesian analysis reported by \citeauthor{chávez_de_la_peña_vandekerckhove:in_press:hierarchical} (\citeyearNP{chávez_de_la_peña_vandekerckhove:in_press:hierarchical}, Appendix~D) took approximately 3,800~ms. The sub-millisecond computation times of the EZ bootstrap demonstrate that the method makes real-time\footnote{We have a strict definition of ``real-time'' as less than the duration of one display frame in a typical experimental setup -- usually at most 1/120th of a second or 8.33~ms.} uncertainty quantification feasible where traditional methods would be prohibitively slow.


\section{Application to the circular diffusion model}

The circular diffusion model \cite{Smith2016} extends the diffusion model to circular decision spaces, where evidence accumulation occurs in a two-dimensional disk rather than along a one-dimensional line. The EZ version of the circular diffusion model (EZ-CDM; \citeNP{QarehdaghiRad2024}) provides analytical expressions that map four summary statistics---circular mean of choice angles, circular variance of choice angles, mean response time, and response time variance---to four cognitive parameters: drift angle, drift magnitude, boundary radius, and non-decision time. This makes the EZ-CDM framework a synthetic likelihood model: it provides a likelihood function based on summary statistics rather than raw trial data.

\subsection{The EZ circular diffusion model}

The EZ circular diffusion model provides analytical expressions for four summary statistics from a set of $n$ circular decision trials.  Let $\MCA{}$ denote the observed circular mean of choice angles, $\VCA{}$ the observed circular variance of choice angles, $\MRT{}$ the observed mean response time, and $\VRT{}$ the observed response time variance. The model's four parameters are the drift angle $\driftangle$, drift magnitude $\driftmag$, boundary radius $\boundaryradius$, and non-decision time $\ndtcdm$.
This allows us to define forward equations 
$
f_\text{CDM}:
\left(\driftangle, \driftmag, \boundaryradius, \ndtcdm\right)
\mapsto
\left(\MCA{}, \VCA{}, \MRT{}, \VRT{}\right)
$ and inverse equations 
$
i_\text{CDM}:
\left(\MCA{}, \VCA{}, \MRT{}, \VRT{}\right)
\mapsto
\left(\driftangle, \driftmag, \boundaryradius, \ndtcdm\right)
$.

The forward equations $f_\text{CDM}: \left(\driftangle, \driftmag, \boundaryradius, \ndtcdm\right) \mapsto \left(\MCA{}, \VCA{}, \MRT{}, \VRT{}\right)$ map parameters to summary statistics. The choice-angle distribution is von Mises with mean $\mu = \driftangle$ and concentration $\kappa = \boundaryradius\driftmag$.
The circular mean and variance are then:
\begin{eqnarray}
\MCA{} &=& \driftangle \nonumber\\
\VCA{} &=& 1 - R \nonumber
\end{eqnarray}
where $R = {I_1(\kappa)}/{I_0(\kappa)}$ and $I_0$ and $I_1$ are modified Bessel functions of the first kind. The response time moments are:
\begin{eqnarray}
\MRT{} &=& \ndtcdm + \frac{\boundaryradius}{\driftmag} R \nonumber\\
\VRT{} &=& \frac{\boundaryradius^2}{\driftmag^2} R^2 + \frac{2\boundaryradius}{\driftmag^3} R - \boundaryradius^2 \nonumber
\end{eqnarray}

The inverse equations $i_\text{CDM}: \left(\MCA{}, \VCA{}, \MRT{}, \VRT{}\right) \mapsto \left(\driftangle, \driftmag, \boundaryradius, \ndtcdm\right)$ map summary statistics to parameters. The estimation proceeds in four steps:
\begin{eqnarray}
\hat{\driftangle} &=& \MCA{} \nonumber\\
R &=& 1 - \VCA{}, \quad \text{solve } \frac{I_1(\kappa)}{I_0(\kappa)} = R \text{ for } \kappa \nonumber\\
\hat{\driftmag}^2 &=& \frac{-\kappa^2 + \sqrt{\kappa^4 + 4\VRT{}(\kappa^2 R^2 + 2\kappa R)}}{2\VRT{}}, \quad \hat{\boundaryradius} = \frac{\kappa}{\hat{\driftmag}} \nonumber\\
\hat{\ndtcdm} &=& \MRT{} - \frac{\hat{\boundaryradius}}{\hat{\driftmag}}R \nonumber
\end{eqnarray}
where the solution for $\hat{\driftmag}^2$ is obtained by solving the quadratic equation derived from the forward equation for $\VRT{}$.
Solving $R = I_1(k)/I_0(k)$ for $k$ is faciliated by the piecewise estimator of \citeA[see pp.~88-93 for further detail]{Fisher1993}: $$k \approx \begin{cases} 2R + R^{3} + \tfrac{5}{6}R^{5}, & R < 0.53,\\ -0.4 + 1.39R + \dfrac{0.43}{1-R}, & 0.53 \le R < 0.85,\\ \dfrac{1}{R^{3} - 4R^{2} + 3R}, & R \ge 0.85.\end{cases}$$
As an excellent alternative, \citeA{QarehdaghiRad2024} suggest performing a single step of Newton-Raphson optimization. This approach typically provides higher accuracy than the piecewise approximation but requires extra computational effort.

% \textit{Sampling distributions.}
The sampling distributions of the four summary statistics determine the variability and robustness of EZ-CDM parameter estimates. Under the CDM, choice angles follow a von Mises distribution with concentration $\kappa = \boundaryradius\driftmag$, and response times follow the first-passage-time distribution of a two-dimensional Bessel process with drift.

For the circular mean of choice angles (MCA), the angular error $\widehat{\driftangle} - \driftangle$ is approximately normally distributed for large $N$ \cite{Fisher1993}, and so:
\begin{equation}
\widehat{\driftangle} \sim \mathcal{N}\left(\driftangle, \frac{1}{N\kappa R}\right). \nonumber
\end{equation}
The variance decreases rapidly with concentration $\kappa$, making estimation more difficult when drift magnitude or boundary radius are small.

For the circular variance ($\VCA{}$), the mean resultant length $L = \sqrt{\left(\frac{1}{N}\sum_{i=1}^{N} \cos(\theta_i)\right)^2 + \left(\frac{1}{N}\sum_{i=1}^{N} \sin(\theta_i)\right)^2}$ is approximately normal for moderate-to-large $N$, with mean $R$ and variance $\frac{1 - R^2}{N}$ \cite{Fisher1993}. Consequently, $\widehat{\VCA{}} = 1 - R$ is also asymptotically normal with the same variance. The circular variance is only weakly sensitive to outliers in angle (since angles wrap on the circle). However, parameter estimation becomes unstable when $R$ is near zero (because then the data are essentially uniform on the circle and provide limited information about the concentration parameter $\kappa$).

For mean response time ($\MRT{}$), the central limit theorem applies:
\begin{equation}
\widehat{\MRT{}} \sim \mathcal{N}\left(\MRT{}, \frac{\VRT{}}{N}\right) \nonumber
\end{equation}
Mean RT is robust for large $N$ but sensitive to RT contaminants.

For response time variance ($\VRT{}$), the sample variance is asymptotically normal but with slower convergence:
\begin{equation}
\widehat{\VRT{}} \sim \mathcal{N}\left(\VRT{}, \frac{m_4 - \VRT{}^2}{N}\right) \nonumber 
\end{equation}
where $m_4$ is the fourth central moment of the RT distribution. This statistic is the most contamination-sensitive in EZ-CDM, as a small number of extreme RTs can cause large upward shifts (see \citeNP{chávez_de_la_peña_etal:preprint:hierarchical}, and \citeNP{QarehdaghiRad2024}, for approaches using robust summary statistics).

The EZ bootstrap procedure can be applied to EZ-CDM by resampling these statistics from their asymptotic distributions. The angle-based statistics ($\MCA{}$ and $\VCA{}$) have well-behaved sampling distributions, while the RT-based statistics ($\MRT{}$ and $\VRT{}$) introduce most of the sampling variability in the inversion system, especially when solving for drift magnitude and boundary radius.

\subsection{Simulation studies}

We again present two calibration studies to validate the EZ bootstrap method for the EZ circular diffusion model.  The target of the study was again the calibration of the uncertainty quantification in terms of credibility interval coverage (95\% credibility).

\subsubsection{Study 1: Single condition}
In Study 1, we evaluated coverage for the regular EZ circular diffusion model with a single condition. We simulated $n = 100$ trials from true parameters $\driftangle = 0.0$, $\driftmag = 1.0$, $\boundaryradius = 1.0$, and $\ndtcdm = 0.2$. For each of 10,000 simulation repetitions, we computed credible intervals at the 95\% confidence level using the EZ bootstrap method (with $1000$ bootstrap samples). 

% \textit{Results.}
We evaluated coverage at the 95\% confidence level across 10,000 simulation repetitions. The EZ bootstrap method achieved excellent calibration. Coverage rates were 93.9\%, 94.3\%, 95.2\%, and 94.8\% for drift angle, drift magnitude, boundary radius, and non-decision time, respectively---all within expected sampling variation of the nominal 95\% level. When considering all four parameters simultaneously, the coverage rate was 82.4\%, closely matching the expected value ($0.95^4 \approx 0.815$).

\subsubsection{Study 2: Design matrix extension}
In Study 2, we evaluated coverage for the EZ circular diffusion model with a design matrix extension. We simulated data from four conditions using a design matrix with three regression weights for each parameter type (drift angle, drift magnitude, boundary radius, and non-decision time), resulting in 12 beta weight parameters total. The true beta weights were: drift angle weights $[0.0, 0.5, 1.0]$, drift magnitude weights $[0.5, 0.7, 0.9]$, boundary radius weights $[0.8, 1.0, 1.2]$, and non-decision time weights $[0.15, 0.2, 0.25]$ \cite{villarreal_etal:2024:graphical}.

The design matrices for drift angle, drift magnitude, boundary radius, and non-decision time parameters were:
$$
X_{\driftangle} = \begin{pmatrix}
1 & 0 & 0 \\
0 & 1 & 0 \\
0 & 0 & 1 \\
0 & 1 & 0
\end{pmatrix}\!\!,\,\,\,
X_{\driftmag} = \begin{pmatrix}
1 & 0 & 0 \\
0 & 1 & 0 \\
0 & 0 & 1 \\
0 & 1 & 0
\end{pmatrix}\!\!,$$
$$X_{\boundaryradius} = \begin{pmatrix}
1 & 0 & 0 \\
0 & 1 & 0 \\
0 & 0 & 1 \\
1 & 0 & 0
\end{pmatrix}\!\!,\,\,\,
X_{\ndtcdm} = \begin{pmatrix}
1 & 0 & 0 \\
0 & 1 & 0 \\
0 & 0 & 1 \\
0 & 0 & 1
\end{pmatrix} 
$$
where each row corresponds to a condition and each column corresponds to a regression weight.
Each condition had $n = 100$ trials. We conducted 10,000 simulation repetitions, computing credible intervals at the 95\% confidence level using EZ bootstrap (with $1000$ bootstrap samples).

The method maintained excellent calibration: For drift angle weights, coverage rates were 89.5\%, 94.5\%, and 94.2\% across the three weights. For drift magnitude weights, coverage rates were 94.5\%, 94.4\%, and 89.3\%. For boundary radius weights, coverage rates were 95.8\%, 94.8\%, and 93.0\%. For non-decision time weights, coverage rates were 95.8\%, 94.6\%, and 95.7\%. All coverage rates were within expected sampling variation of the nominal 95\% level.

\subsubsection{Computational performance}
For Study~1, the method required a mean of only 0.33~ms per estimation (SD = 0.03~ms), enabling rapid uncertainty quantification. For Study~2, with the added complexity of the design matrix, the method required a mean of 1.26~ms per estimation (SD = 0.19~ms). 

\section{Other models}

The EZ bootstrap method generalizes to any model in the EZ class -- that is, any model with analytical forward and inverse equations smoothly mapping parameters to summary statistics. Here we briefly describe one additional example: the one-high-threshold source-monitoring {multinomial processing tree (MPT)} model.

\subsection{One-high-threshold source-monitoring model}

The one-high-threshold source-monitoring MPT model \cite{BatchelderRiefer1990,BatchelderRiefer1999} is a multinomial processing tree model for source memory that allows participants to identify items as coming from Source~A, Source~B, or as new. The model has four parameters: detection probability $\Detect$, source discrimination probability $\Discrim$, criterion for new items $\Criterion$, and guessing probability $\Guess$.

The forward equations map parameters to observed response proportions. For old items from Source~A, the forward equations are:
\begin{eqnarray}
\AAstat &=& \Detect\Discrim + (1-\Detect)\Guess \nonumber\\
\BAstat &=& \Detect(1-\Discrim) + (1-\Detect)(1-\Guess) \nonumber
\end{eqnarray}
where $\AAstat = P(\text{``A''} \mid \text{old from A})$ and $\BAstat = P(\text{``B''} \mid \text{old from A})$. Analogous expressions hold for old items from Source~B:
\begin{eqnarray}
\ABstat &=& \Detect(1-\Discrim) + (1-\Detect)\Guess \nonumber\\
\BBstat &=& \Detect\Discrim + (1-\Detect)(1-\Guess) \nonumber
\end{eqnarray}
where $\ABstat = P(\text{``A''} \mid \text{old from B})$ and $\BBstat = P(\text{``B''} \mid \text{old from B})$. For new items, the model specifies:
\begin{eqnarray}
\NNstat &=& \Criterion \nonumber\\
\ANstat &=& (1-\Criterion)\Guess \nonumber\\
\BNstat &=& (1-\Criterion)(1-\Guess) \nonumber
\end{eqnarray}
where $\NNstat = P(\text{``new''} \mid \text{new})$, $\ANstat = P(\text{``A''} \mid \text{new})$, and $\BNstat = P(\text{``B''} \mid \text{new})$.
The inverse equations provide closed-form parameter estimates:
\begin{eqnarray}
\Criterion &=& \NNstat \nonumber\\
\Guess &=& \frac{\ANstat}{1-\NNstat} \nonumber\\
\Detect &=& \frac{\AAstat + \ABstat - 2\Guess}{1-2\Guess} \nonumber\\
\Discrim &=& \frac{\AAstat - (1-\Detect)\Guess}{\Detect} \nonumber
\end{eqnarray}
These linear relationships allow closed-form inversion, illustrating a simple but nontrivial MPT for which both forward and inverse transformations are available in closed form.

The sampling distributions for MPT models are multinomial: the observed counts in each response category follow a multinomial distribution with probabilities given by the forward equations. For old items from Source~A, the counts $(n_{\AAstat}, n_{\BAstat})$ follow:
\begin{equation}
(n_{\AAstat}, n_{\BAstat}) \sim \text{Multinomial}(n_A, (\AAstat, \BAstat)) \nonumber
\end{equation}
where $n_A$ is the number of old items from Source~A and $\AAstat + \BAstat = 1$. Similarly, for old items from Source~B:
\begin{equation}
(n_{\ABstat}, n_{\BBstat}) \sim \text{Multinomial}(n_B, (\ABstat, \BBstat)) \nonumber
\end{equation}
where $n_B$ is the number of old items from Source~B and $\ABstat + \BBstat = 1$. For new items:
\begin{equation}
(n_{\NNstat}, n_{\ANstat}, n_{\BNstat}) \sim \text{Multinomial}(n_N, (\NNstat, \ANstat, \BNstat)) \nonumber
\end{equation}
where $n_N$ is the number of new-item trials and $\NNstat + \ANstat + \BNstat = 1$. The EZ bootstrap procedure can be applied by resampling from these multinomial distributions and transforming through the inverse equations to obtain bootstrap parameter estimates.

We provide this solvable MPT example only as an illustration. The class of MPT models with tractable analytical inverses is rather narrow and does not reflect the complexity of most MPT applications, where estimation usually requires numerical optimization or Bayesian inference \cite{matzke_etal:2018:nonstandard}.

\section{Discussion}

The EZ bootstrap method provides fast uncertainty quantification. It maintains proper calibration while achieving dramatic speed improvements over previous methods such as the Bayesian implementation by \citeA{chávez_de_la_peña_vandekerckhove:in_press:hierarchical} -- the method achieves coverage rates that closely match nominal levels for both simple single-condition models and more complex design matrix extensions.

The method's success relies on the synthetic likelihood structure of EZ models. By operating directly on summary statistics with known sampling distributions, we avoid the computational cost of raw data resampling while preserving statistical validity. The transformation-of-variables approach generalizes to any model in the EZ class -- that is, any model with invertible mappings from parameters to summary statistics. The calibration results illustrate that our method maintains proper statistical properties while delivering computational efficiency.


\subsection{Limitations}

{The bootstrap method has a few limitations worth pointing out. First, the method is limited to the class of EZ models and their design matrix extensions. While identifiable models will typically \textit{have} both the forward and inverse equations, the equations are often not known or are mathematically inconvenient. Second, the method requires that summary statistics have known or approximately known sampling distributions. Approximations such as the normal distribution for mean RT and the Gamma for variance are asymptotically valid but may be less accurate for very small samples ($n < 30$) or when RT distributions are highly skewed, but good approximations of sampling distributions are known for very many summary statistics. Third, we assume approximate independence between summary statistics. While this holds asymptotically and is approximately valid for sufficiently large samples, correlations may be non-negligible for very small samples or models beyond our consideration.}


\subsection{Future directions}

Several extensions are possible. The method could be extended to yet more models with known mappings from parameters to summary statistics. Hierarchical extensions could enable multi-level modeling with summary statistics. Integration with experimental platforms could enable real-time semi-Bayesian design optimization and computerized adaptive tests.

Perhaps most promisingly, implicit mappings from parameters to summary statistics exist even when explicit equations are unknown. Techniques from statistical learning, such as normalizing flow neural networks, could be used to approximate these mappings for use in the synthetic bootstrap method, extending the approach to a broader class of models.


\section{Computational reproducibility}

All code associated with this project is available via GitHub. The simulation studies reported in this paper can be reproduced using the C implementation available in the GitHub repository \texttt{joachimvandekerckhove/ez-bootstrap}. The implementation requires the GNU Scientific Library (GSL) for random number generation and statistical distributions.

To reproduce the simulation studies, first
(1)~clone the repository, then
(2)~install dependencies if needed (e.g., install \texttt{libgsl-dev} to obtain GSL development libraries on Debian/Ubuntu systems), and 
(3)~run the simulation script
\texttt{./run\_all.sh}.

The script executes both simulation studies for the EZ-DDM and EZ-CDM with the following editable parameters: 10,000 simulation repetitions, sample size $n = 100$ trials per condition, and 1,000 bootstrap samples per repetition. The simulation outputs include coverage rates for each parameter (or beta weight) and computational performance metrics.

{A Python implementation of the EZ-DDM is also available as part of the larger EZAS package (\url{https://github.com/joachimvandekerckhove/ezas}), which uses the same method but is not as fast as the C implementation.}


\ifdefined\apaformat

\newcommand{\dec}[2]{\noindent\textbf{#1} --- #2}

\section{Declarations}

\dec{Funding}{This work was supported by the National Research Council.}
   
\dec{Competing interests}{The authors report no competing interest.}

\dec{Ethics approval}{N/A}

\dec{Consent to participate}{N/A}

\dec{Consent for publication}{N/A}

\dec{Availability of data and materials}{N/A}

\dec{Code availability}{See Section ``Computational reproducibility.''}

\fi


\bibliography{includes/references,includes/cidlab}\null

\subsection{Disclaimer}
The views expressed are those of the authors and do not necessarily reflect the official policy or position of the Department of the Air Force, the Department of Defense, or the U.S. Government. 


% \appendix
% \ifdefined\apaformat
% \section{Reproducing the simulation studies}
% \else
% \section{Appendix: Reproducing the simulation studies}
% \fi

\end{document}
